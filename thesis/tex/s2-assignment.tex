{
Студенту Прекель Владиславу Артемовичу

Группа: КИ18-16б Направление (специальность): 09.03.04 Программная~инженерия

Тема выпускной квалификационной работы: Система физического моделирования на основе априорного подхода обнаружения столкновений

Утверждена приказом по университету №~6060/с от 21.04.22

Руководитель ВКР: Ю.~Ю.~Якунин, заведующий Базовой кафедрой интеллектуальных систем управления, кандидат технических наук
Института космических и информационных технологий

Исходные данные для ВКР: описание предметной области, Интернет ресурсы

Перечень разделов ВКР: теоретические сведения, проектирование, программная реализация, примеры использования и перспективы

Перечень графического материала ВКР: презентация

\vfill

\setlength{\parindent}{0cm}

Руководитель ВКР
\hfill
\uline{\hspace{10.5ex}}
\hspace{6ex}
\uline{Ю.~Ю.~Якунин}

\vspace{-4pt}

\hfill
{\footnotesize подпись}
\hspace{8.5ex}
{\footnotesize инициалы и фамилия}

Задание принял к исполнению~
\hfill
\uline{\hspace{18.5ex}В.~А.~Прекель}

\vspace{-4pt}

\hfill
{\footnotesize подпись, инициалы и фамилия студента}

\vspace{1em}

\hfill <<\uline{\hspace{3ex}}>> \uline{\hspace{9ex}} 2022 г.
}

\thispagestyle{empty}

\clearpage
