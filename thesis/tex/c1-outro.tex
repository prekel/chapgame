\Anonchapter{Заключение}

В ходе выполнения выпускной квалификационной работы бакалавра был разработан
физический движок, использующий априорный подход для обнаружения столкновений.

Были выполнены все поставленные задачи, а именно
в разделе~\ref{chapter-theory} определена модель и систематизирована математическая база, требующаяся для реализации движка;
в разделе~\ref{chapter-design} проведён обзор используемых технологий при разработке;
в разделе~\ref{chapter-impl} описана программная реализация физического движка и интерактивной демонстрацию его работы;
в разделе~\ref{chapter-usage} продемонстрирована работа движка на примерах и обозначены возможности развития.

При выполнении работы были применены компетенции, полученные на дисциплинах
<<Алгебра и геометрия>>,
<<Математический анализ>>,
<<Дискретная математика>>,
<<Физика>>,
<<Математическая логика и теория алгоритмов>>,
<<Основы функционального программирования>>,
<<Разработка web-приложений>>,
<<Алгоритмы и структуры данных>>
и~др.

Репозиторий с исходным кодом выложен на GitHub по адресу
\underline{\smash{\href{https://github.com/prekel/chapgame}{https://github.com/prekel/chapgame}}}.
Интерактивная демонстрация работы движка развёрнута на GitHub Pages по адресу
\underline{\smash{\href{https://prekel.github.io/chapgame}{https://prekel.github.io/chapgame}}}.
