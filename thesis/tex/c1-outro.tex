\Anonchapter{Заключение}

В ходе выполнения бакалаврской работе был разработан
физический движок, использующий априорный подход для обнаружения столкновений.
Были выполнены все поставленные задачи, а именно
в главе~\ref{chapter-theory} определена модель и систематизирована математическая база, требующаяся для реализации движка;
в главе~\ref{chapter-design} проведён обзор используемых технологий при разработке;
в главе~\ref{chapter-impl} описана программная реализация физического движка и интерактивной демонстрацию его работы;
в главе~\ref{chapter-usage} продемонстрирована работа движка на примерах и обозначены возможности развития.
Репозиторий с исходным кодом выложен на GitHub по адресу
\underline{\smash{\href{https://github.com/prekel/chapgame}{https://github.com/prekel/chapgame}}}.
Интерактивная демонстрация работы движка развёрнута на GitHub Pages по адресу
\underline{\smash{\href{https://prekel.github.io/chapgame}{https://prekel.github.io/chapgame}}}.

\TODO
