\documentclass[xetex,aspectratio=43]{beamer}
\usepackage{polyglossia}
\usepackage{fontspec}
\usepackage{xunicode} 

\usepackage{graphicx}
\graphicspath{ {../images/} }

\usepackage{listings}

\defaultfontfeatures{Ligatures=TeX}
\setmainfont{Times New Roman}
\setdefaultlanguage{russian}
\setmonofont{Courier New} 

\newfontfamily{\cyrillicfont}{Times New Roman} 
\newfontfamily{\cyrillicfontrm}{Times New Roman}
\newfontfamily{\cyrillicfonttt}{Courier New}
\newfontfamily{\cyrillicfontsf}{Arial}

% \usetheme{Pittsburgh}
\usetheme{default}
\usecolortheme{seahorse}
\setbeamersize{text margin left=7mm,text margin right=7mm} 

\title{Выпуская квалификационная работа}
\subtitle{Система физического моделирования на основе априорного подхода обнаружения столкновений}
\author{Владислав Прекель}
\institute{ИКИТ СФУ\\КИ18-16б}
\date{Красноярск\\2 июня 2022 г.}

\usepackage[fleqn]{mathtools}
\usepackage{unicode-math}

\newenvironment{Underequation}{
    \small
    \noindent
    где
    \hspace{-1.45ex}
    \setlength{\parindent}{3.5ex}
}{}

\begin{document}
\begin{frame}
    \titlepage
\end{frame}

\begin{frame}
    \frametitle{Модель}

    \includegraphics[height=6cm]{body_init}

\end{frame}

\begin{frame}
    \frametitle{Формулы равноускоренного движения}
    Формулы для скорости~(\ref{velocityvec_1}) и положения тела~(\ref{r_x_1}):
    % , ускорения~(\ref{})

    \begin{equation}\label{velocityvec_1}
        \vec{v}(t) = \vec{v_0} + \vec{a}t
    \end{equation}

    \begin{equation}\label{r_x_1}
        x(t) = x_0 + {v_0}_x t + \frac{a_x t^2}{2},
        y(t) = y_0 + {v_0}_y t + \frac{a_y t^2}{2}
    \end{equation}

    % \begin{equation}
    %     \vec{a} = \frac{\vec{F_{\text{тр.}}}}{m} = -\frac{\vec{v_0} \mu g}{\left|\vec{v_0}\right| m}
    % \end{equation}

    \begin{Underequation}
        \(\vec{v}(t)\)~-- вектор скорости тела в момент времени \(t\);

        \(\vec{v_0}\)~-- вектор начальной скорости тела;

        \(\vec{a}\)~-- вектор ускорения тела;

        \(x(t)\)~-- координата тела в момент времени \(t\) по оси \(X\);

        \(x_0\)~-- координата начального положения тела по оси \(X\);

        \(y(t)\)~-- координата тела в момент времени \(t\) по оси \(Y\);

        \(y_0\)~-- координата начального положения тела по оси \(Y\);

        \(a_x\)~-- проекция вектора ускорения тела \(\vec{a}\) на ось \(X\);

        \(a_y\)~-- проекция вектора ускорения тела \(\vec{a}\) на ось \(Y\).
    \end{Underequation}
\end{frame}

\begin{frame}
    \frametitle{Тела столкнулись}

    \includegraphics[height=6cm]{body_collision}

\end{frame}

\begin{frame}
    \frametitle{Апостериорный подход}

    \includegraphics[height=6cm]{body_aposteriori}
\end{frame}

\begin{frame}
    \frametitle{Априорный подход}

    Основан на том, что можно найти время столкновения через уравнение (\ref{distance_equation}):

    \begin{equation}\label{distance_equation}
        distance(t) = r_1 + r_2
    \end{equation}

    \begin{Underequation}
        \(distance(t)\)~-- расстояние между центрами двух тел в момент времени \(t\);

        \(r_1\)~-- радиус первого тела;

        \(r_2\)~-- радиус второго тела.
    \end{Underequation}

\end{frame}

\begin{frame}
    \frametitle{Цель работы}

    \textbf{Целью выпускной квалификационной работы} является
    разработка физического движка, использующего априорный подход для обнаружения столкновений.

\end{frame}

\begin{frame}
    \frametitle{Уравнение обнаружения столкновения двух тел}

    \begin{equation}\label{bodybodyoft}
        \sqrt{(x_1(t) - x_2(t))^2 + (y_1(t) - y_2(t))^2} = r_1 + r_2
    \end{equation}

    \begin{Underequation}
        \(x_1(t)\)~-- координата первого тела в момент времени \(t\) по оси \(X\);

        \(x_2(t)\)~-- координата второго тела в момент времени \(t\) по оси \(X\);

        \(y_1(t)\)~-- координата первого тела в момент времени \(t\) по оси \(Y\);

        \(y_2(t)\)~-- координата второго тела в момент времени \(t\) по оси \(Y\);

        \(r_1\)~-- радиус первого тела;

        \(r_2\)~-- радиус второго тела.
    \end{Underequation}

\end{frame}

\begin{frame}
    \vspace{-1.5em}
    \begin{align}
        \frac{{a_x}_1^2 - {a_x}_2^2 + {a_y}_1^2 - {a_y}_2^2  }{4}                                                                    & t^4~+  \nonumber                        \\
        + ({{v_0}_x}_1 {a_x}_1 - {{v_0}_x}_2 {a_x}_2 + {{v_0}_y}_1 {a_y}_1 - {{v_0}_y}_2 {a_y}_2)                                    & t^3~+  \nonumber                        \\
        + ({x_0}_1 {a_x}_1 - {x_0}_2 {a_x}_2 + {y_0}_1 {a_y}_1 - {y_0}_2 {a_y}_2)                                                    & t^2~+  \nonumber                        \\
        + 2 ({x_0}_1 {{v_0}_x}_1 - {x_0}_2 {{v_0}_x}_2 + {y_0}_1 {{v_0}_y}_1 - {y_0}_2 {{v_0}_y}_2)                                  & t~+                                     \\
        \hspace{-1em}+ {{v_0}_x}_1^2 + {x_0}_1^2 - {{v_0}_x}_2^2 - {x_0}_2^2 + {{v_0}_y}_1^2 + {y_0}_1^2 - {{v_0}_y}_2^2 - {y_0}_2^2 & - r_1^2 - 2r_1 r_2 - r_2^2 = 0\nonumber
    \end{align}

    {\small
    \begin{Underequation}
        \({x_0}_1\)~-- начальная координата первого тела по оси \(X\);

        \({x_0}_2\)~-- начальная координата второго тела по оси \(X\);

        \({y_0}_1\)~-- начальная координата первого тела по оси \(Y\);

        \({y_0}_2\)~-- начальная координата второго тела по оси \(Y\);

        \({{v_0}_x}_1\)~-- проекция вектора начальной скорости I тела на ось \(X\);

        \({{v_0}_y}_1\)~-- проекция вектора начальной скорости I тела на ось \(Y\);

        \({{v_0}_x}_2\)~-- проекция вектора начальной скорости II тела на ось \(X\);

        \({{v_0}_y}_2\)~-- проекция вектора начальной скорости II тела на ось \(Y\);

        \({a_x}_1\)~-- проекция вектора ускорения I тела на ось \(X\);

        \({a_y}_1\)~-- проекция вектора ускорения I тела на ось \(Y\);

        \({a_x}_2\)~-- проекция вектора ускорения II тела на ось \(X\);

        \({a_y}_2\)~-- проекция вектора ускорения II тела на ось \(Y\).
    \end{Underequation}}
\end{frame}

\begin{frame}
    \frametitle{Метод численного решения алгебрических уравнений}

    \centerline{\includegraphics[height=4cm]{equation_deg3_plot}}

    \small

    Рассмотрим на примере уравнения \(x^3-2x^2-x+2=0\) (красное).

    Уравнение производной \(3x^2-4x-1=0\) (синее).

    Его корни: \(x_{A,B} = \frac{4 \pm \sqrt{28}}{6}\).

    Тогда, корни исходного уравнения можно найти методом бисекции:

    \(x_1\) на промежутке \((-\infty; \frac{4 - \sqrt{28}}{6}]\), будет равен \(-1\)\\
    \(x_2\) на промежутке \([\frac{4 - \sqrt{28}}{6}; \frac{4 + \sqrt{28}}{6}]\), будет равен \(1\)\\
    \(x_3\) на промежутке \([\frac{4 + \sqrt{28}}{6}; +\infty)\), будет равен \(2\)

\end{frame}

\begin{frame}
    \frametitle{Выбор нужного корня}

    \includegraphics[height=8cm]{body_second_root}
\end{frame}

\begin{frame}
    \frametitle{Уравнение обнаружения столкновения с точкой}

    \begin{equation}\label{bodypointcollisioncoords}
        \sqrt{(x(t) - p_x)^2 + (y(t) - p_y)^2} = r
    \end{equation}

    \begin{Underequation}
        \(x(t)\)~-- координата положения тела по оси \(X\);

        \(y(t)\)~-- координата положения тела по оси \(Y\);

        \(r\)~-- радиус тела;

        \(p_x\)~-- координата точки по оси \(X\);

        \(p_y\)~-- координата точки по оси \(Y\).
    \end{Underequation}
\end{frame}

\begin{frame}
    \frametitle{Уравнение обнаружения столкновения с прямой}

    \begin{equation}\label{bodylinecolision}
        \frac{\left|A x(t) + B y(t) + C\right|}{\sqrt{A^2 + B^2}} = r
    \end{equation}

    \begin{Underequation}
        \(A\),~\(B\),~\(C\)~-- коэффициенты общего уравнения прямой;

        \(r\)~-- радиус тела;

        \(x(t)\),~\(y(t)\)~-- координаты тела в момент времени \(t\).
    \end{Underequation}

\end{frame}

\begin{frame}
    \frametitle{Обработка ударов}

    \begin{equation}\label{collisionhandle}
        \vec{v_1^\prime} = \vec{v_1} - \frac{2 m_2}{m_1 + m_2}
        \frac{\langle \vec{v_1} - \vec{v_2}, \vec{r_1} - \vec{r_2} \rangle }{\left| \vec{r_1} - \vec{r_2} \right|^2}
        (\vec{r_1} - \vec{r_2})
    \end{equation}

    \begin{Underequation}
        \(\langle , \rangle\)~-- скалярное произведение векторов;

        \(\vec{v_1^\prime}\)~-- вектор скорости первого тела после удара;

        \(\vec{v_1}\)~-- вектор скорости первого тела до удара;

        \(\vec{v_2}\)~-- вектор скорости второго тела до удара;

        \(\vec{r_1}\)~-- радиус-вектор положения первого тела;

        \(\vec{r_2}\)~-- радиус-вектор положения второго тела;

        \(m_1\)~-- масса первого тела;

        \(m_2\)~-- масса второго тела.
    \end{Underequation}

\end{frame}

\begin{frame}
    \frametitle{Использованные технологии}

    \begin{itemize}
        \item OCaml~-- язык программирования;
        \item Js\_of\_ocaml~-- компилятор OCaml в JavaScript;
        \item Lwt~-- библиотека для конкурентного программирования;
        \item Core~-- стандартная библиотека;
        \item Dream~-- web-фреймворк;
        \item ppx\_inline\_test,~ppx\_expect~-- библиотеки юнит-тестирования;
        \item Sexplib~-- библиотека для сериализации и десериализации S-выражений;
        \item Bulma~-- CSS-фреймворк;
        \item Dune, opam~-- система сборки и пакетный менеджер;
        \item VS Code, OCaml Platform~-- среда разработки и плагин для работы с OCaml.
    \end{itemize}


\end{frame}

\begin{frame}[fragile]
    \frametitle{Реализация движка}

    \footnotesize
    \begin{lstlisting}[language=ml]
S.Model.init ~g:1.
|> S.Engine.recv ~action:{ time = 0.
        ; action =
            AddBody { id = Some id1; x0 = 350.; y0 = 200.
                    ; r = 100.; mu = 1.; m = 10. }
        ; until = { timespan = Some 0.; quantity = None }}
|> S.Engine.recv ~action:{ time = 0.
        ; action =
            AddBody { id = Some id2; x0 = 700.; y0 = 200.
                    ; r = 100.; mu = 1.; m = 10. }
        ; until = { timespan = Some 0.; quantity = None }}
|> S.Engine.recv ~action:{ time = 0.
        ; action = GiveVelocity { id = id2; v0 = -100., 0. }
        ; until = { timespan = None; quantity = None }}
    \end{lstlisting}

\end{frame}

\begin{frame}
    \frametitle{Интерактивная демострация возможностей движка}

    \includegraphics[height=8cm]{chapgame_1}

\end{frame}

\begin{frame}
    {\color{blue}\url{https://prekel.github.io/chapgame/}}

    \includegraphics[height=8cm]{chapgame_2}

\end{frame}

\begin{frame}
    \frametitle{Решённые задачи}

    \begin{itemize}
        \item определена модель и математическая база, требующуюся для моделирования;
              %\item проведён обзор используемых технологий при разработке (второй раздел ВКР);
        \item программно реализован физический движок и интерактивная демонстрация его работы.
    \end{itemize}
\end{frame}

\begin{frame}
    \titlepage
\end{frame}

\end{document}
