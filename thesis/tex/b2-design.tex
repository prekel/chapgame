\chapter{Проектирование}\label{chapter-design}

Кроме создания физического движка, важно продемонстрировать его работу.
Желательно в интерактивном режиме. В XXI веке сложилась ситуация, что веб-приложения
обладают преимуществом, таким что для того чтобы им воспользовался пользователь, достаточно
браузера и просто перейти по ссылке.
Поэтому, решено сделать Single Page Application
(т.е. веб-приложение, которое загружает только один веб-документ,
а затем обновляет содержимое тела этого документа с помощью JavaScript API,
когда необходимо показать другое содержимое~\cite{mdn-spa}),
которое в интерактивном режиме отображает состояние модели (описанной в пункте~\ref{model_1_3}),
меняющееся с течением времени.

Так как создаваемый движок будет способен работать в реальном времени, существует возможность
создать многопользовательский режим, для которого потребуется серверная часть,
на которую перенесётся работа движка, а клиенты будут получать лишь изменения в модели (подробнее в~\ref{clientonlineimpl} \TODO).
Соответственно, браузерное интерактивное SPA приложение с графическим интерфейсом назовём клиентской часть.

Так как мы хотим, чтобы движок работал и на клиентской части, и на серверной, следует выбрать язык программирования,
который позволяет создавать и браузерные приложения, и нативные.

При этом для того чтобы сделать веб-приложение интерактивным, браузеры могут работать лишь с JavaScript
с возможность подключения WebAssembly-модулей. WebAssembly~-- относительно новый способ создавать
приложения на языке, отличным от JavaScript; он представляет из себя байткод стековой машины,
который может быть получен из множества языков~\cite{wasm}, например
C/С++~\cite{emscripten-about}, Rust~\cite{rust-wasm}, C\#~\cite{blazor-ru}, F\#~\cite{fsbolero} и т.д.~\cite{wasm-iwantto}.

Однако существует и другой способ
писать на языке отличном от JavaScript~-- это использования компилятора, который исходный код на нужном языке
компилирует в код на языке JavaScript, пригодный для исполнения в браузере. Поэтому существуют языки программирования,
пригодные для исполнения и в браузере, и на сервере, что приводит к переиспользованию кода и уменьшении времени разработки.
Примеры таких языков: OCaml (подробнее в пункте~\ref{jsoo}), F\#~\cite[с.~48]{dsyme-hopl}, TypeScript и т.д.~\cite{typescript-mayorov}.

При выборе языка программирования важно руководствоваться не только его кроссплатформенностью
(в данном случае кроссплатформенностью можно назвать возможность работать и в браузере, и на сервере),
но ещё и опытом работы с ним; стоит обратить внимание на выразительность, способность к обнаружению ошибок,
производительность, экосистему. В качестве такого языка программирования выбран OCaml.
OCaml является функциональным языком программирования, что можно рассматривать как преимущество
при использовании как инструмента для решения задач численного моделирования~\cite{shutov-haskell}.

\section{Язык программирования OCaml}

OCaml (до 2011 -- Objective Caml~\cite{camlhistory})~-- промышленный язык программирования общего назначения,
в котором особое внимание уделяется выразительности и надёжности~\cite{ocamlorg}. Обладает
мультипарадигменностью, но в первую очередь преподноситься как функциональный язык программирования.

Ключевые достоинства и черты OCaml, согласно~\cite[с.~3]{yaron2011} и~\cite{rwo-prologue}:

\begin{itemize}
      \item \textbf{выразительность}. на OCaml можно создавать более компактные, простые и понятные системы,
            чем на таких языках, как Java или C\#;
      \item \textbf{обнаружение ошибок}. Cуществует удивительно
            широкий круг ошибок, против которых система типов эффективна, включая многие ошибки, которые довольно трудно
            обнаружить с помощью тестирования;
      \item \textbf{производительность}. Производительность OCaml находится на одном уровне или лучше,
            чем у Java, и на расстоянии вытянутой руки от таких языков, как C или C++. В добавок к высококачественной генерации нативного кода,
            OCaml имеет инкрементальный сборщик мусора, который может быть настроен на выполнение небольших
            фрагментов работы за раз, что делает его более подходящим для приложений мягкого реального времени,
            таких как трейдинг;
      \item сборка мусора для автоматического управления памятью, которая сейчас является общей чертой современных языков высокого уровня;
      \item функции первого порядка, которые можно передавать как обычные значения, как в JavaScript, Common Lisp и C\#;
      \item статическая типизация увеличивает производительность и уменьшает число ошибок во время исполнения, как в Java или C\#;
      \item параметрический полиморфизм, позволяющий создавать абстракции, работающие с различными типами данных,
            подобно джереникам из Java, Rust, C\# или шаблонам из C++;
      \item хорошая поддержка иммутабельного программирования, т.е. программирования без деструктивных обновлений в структурах данных.
            Такой подход представлен в традиционных функциональных языках программирования, таких как Scheme, а так же часто встречается
            во всём от распределенных фреймворков для работы с большими данными до UI-тулкитов;
      \item вывод типов, который позволяет не указывать тип каждой переменной в программе;
            Вместо этого, типы выводятся из того, как используется значение. В мейнстримных языках
            такое есть на уровне локальных переменных, например ключевое слово <<var>> в C\# или ключевое слово <<auto>> в C++.
      \item алгебраические типы данных и паттерн-матчинг позволяют определять и манипулировать сложными структурами данных;
            Так же доступны в Scala, Rust, F\#.

\end{itemize}

\section{Обзор экосистемы языка OCaml}

Под экосистемой можно подразумевать совокупность пакетного менеджера, системы сборки, компилятора/интерпретатора, библиотек, фреймворков и т.д.
В XXI веке экосистемы языков программирования развиваются под действием opensource-сообщества.
Качество и размер экосистемы могут зависеть от размеров сообщества. Банально, чем больше разработчиков~-- тем больше библиотек.
OCaml не является мейнстримным языком, т.е. его сообщество относительно мало. Соответственно, и размер экосистемы мал.
Для сравнения: количество пакетов (на момент мая 2022),
доступных через пакетный менеджер opam для OCaml: 3860 единиц~\cite{opam};
через npm для JavaScript: 1,97 млн~\cite{npmjs}.
Несмотря на такой разброс, в экосистеме OCaml достаточно нужного для выполнения проекта.
Все рассмотренные далее библиотеки распространяются с открытым исходным кодом.

\subsection{Компиляторы OCaml в JavaScript}\label{jsoo}

Существует мнение, что JavaScript это Lisp в одежде Си:
несмотря на си-подобный синтаксис, делающий его похожим на обычный процедурный язык,
он больше имеет общего с функциональными языками, такими как Lisp или Scheme,
а не с Си или Java~\cite{crockford}. Благодаря такой семантической схожести и
потребности разработчиков, которые пишут на функциональных языках писать для браузера,
и появляются компиляторы в JavaScript. Примеры таких компиляторов или диалектов для функциональных языков помимо OCaml:
Fable (F\#), ClojureScript (Clojure), RacketScript (Racket), ghcjs (Haskell).
Рассмотрим такие компиляторы для OCaml в порядке появления.

\textbf{Ocamljs}. Ранее существовал компилятор, который для генерации JavaScript
использовал внутреннее <<lambda-представление>> компилятора OCaml~\cite{ocamljs-lambda}.
Так как такое представление является нестабильным (т.е. меняется от релиза к релизу языка OCaml)~\cite{rwo-backend},
такой подход серьёзно усложняет поддержку последних версий языка. В отличие от подхода js\_of\_ocaml,
который кроме того,
помогает добиться большей производительности~\cite[с.~13]{vouillon-jsoo}.

\textbf{Js\_of\_ocaml}. Использует иной подход, основанный на том, что компилятор OCaml
может генерировать кроме нативного кода, ещё и байткод
(генерацию и интерпретацию байткода можно сравнить с тем как работают платформы Java И .NET~--
главное различие, что там используется ещё и JIT-компиляция), что позволяет
интегрироваться с текущей экосистемой~\cite[с.~1]{vouillon-jsoo}.

\textbf{ReScript} (ранее~-- BuckleScript). Использует подобный подход, что и ocamljs~\cite{bobzhang-rawlambda},
но с прицелом на читаемый получаемый код и интеграцией в существующую JavaScript-экосистему~\cite{rescript-introduction}.
При этом у ReScript своя система сборки, а так же устаревшая версия OCaml, что не даёт полноценно
разрабатывать фуллстек приложения на OCaml. Решить эту проблему может форк под название Melange~\cite{melange},
но он находится в состоянии разработки.

Ещё один подход~-- использовать WebAssembly при работе с OCaml, однако
инструментарий для такого подхода сейчас находиться в зачаточном состоянии~\cite{ocaml-wasm}.

\subsection{Стандартные библиотеки}

Встроенная стандартная библиотека OCaml отличается тем, что её фундаментальная роль
служить для самогенерации (bootstrapping) компилятора, поэтому она небольшая и портативная,
что делает её не совсем инструментом общего назначения~\cite{rwo-prologue}.
Поэтому для её замены есть библиотеки Base, Core, Batteries, Containers~\cite{ocamlverse-libraries}.

Base~-- разработанная внутри Jane Street и протестированная в деле реализация стандартной библиотеки~\cite{ocamlverse-libraries}.
Core~-- это надстройка для Base, которая значительно расширяет её функциональность~\cite{janestreet-opensource}.
Библиотека для построения пользовательского интерфейса, выбранная в~\ref{bonsai}, уже имеет в своих зависимостях Core,
поэтому желательно не использовать другую стандартную библиотеку чтобы не создавать лишних зависимостей.

\subsection{Библиотеки конкурентного программирования}

Так как логика работы программ, взаимодействующих с внешним миром, часто предполагает
ожидание: ожидание щелчка мышью, ожидание завершения операции чтения
данных с диска или ожидание освобождения места в выходном сетевом буфере,
требуется использовать приёмы конкурентного программирования~\cite[с.~388]{rwo-ru}.
Один из подходов к организации конкурентного выполнения заключается
в использовании системных потоков выполнения.
Другой подход применяется в однопоточных программах и заключается в
использовании цикла событий, в рамках которого реализуется реакция на внешние
события, такие как тайм-ауты или щелчки мышью, путем вызова функций,
специально зарегистрированных для этого. Этот подход часто используется в языках
программирования, подобных языку JavaScript, имеющему однопоточную среду
выполнения. OCaml так же использует однопоточную среду выполнения,
тоже роднит семантику OCaml с семантикой JavaScript, как указано в пункте~\ref{jsoo}.

Основные библиотеки~-- Lwt и Async. Lwt (<<lightweight threads>>~-- легковесные потоки)
позволяет писать программы с участием потоков в монадическом стиле, что позволяет писать
асинхронный код как обычный OCaml-код~\cite[с.~1]{vouillon-lwt}. Async создавалась с оглядкой на Lwt,
но компании Jane Street, которая создала Async, требовалась другая обработка ошибок,
более лучший контроль над параллелизмом~\cite{announcing-async}. Но при этом, эти две библиотеки
не совместимы между собой, т.е. один проект не может полноценно использовать обе библиотеки
вместе полноценно~\cite{rgrinberg-async}. Из-за этого в экосистеме существует деление среди библиотек,
использующих асинхронность~-- часть библиотек используют Lwt, часть Async. Получается, следует
учитывать что одни библиотеки и фреймворки могут быть не совместимы с другими.
% При этом большинство
% библиотек, использующий Async, написаны так же Jane Street и при создании проекта приходится выбирать,
% использовать стек Jane Street, в котором 

\subsection{Библиотеки для построения пользовательского интерфейса}\label{bonsai}

Рассмотрим бибилиотеки для построения пользовательского интерфейса, совместимые с js\_of\_ocaml.

\textbf{jsoo-react}. Предоставляет биндинги для JavaScript-библиотеки React,
что позволяет использовать React в коде на OCaml~\cite{jsoo-react}.

\textbf{ocaml-vdom}. Реализует Elm-архитектуру~-- функциональный путь для описания пользовательского интерфейса.
Отличается тем, что алгоритм обнаружения изменений в виртуальном DOM, не создаёт структуры данных,
представляющие изменения, а применяет их к реальному DOM <<на лету>>~\cite{lexifi-vdom}.

\textbf{virtual\_dom}. Предоставляет биндинги для JavaScript-библиотеки virtual-dom, позволяющей
производить полноценные операции обнаружения изменений на виртуальном DOM и т.д.~\cite{janestreet-virtualdom}.
Часто не используется непосредственно, а используется через нижеописанные библиотеки.

\textbf{incr\_dom}. Используя библиотеку Incremental для построения самокорректирующихся вычислений,
или вычислений, которые могут быть эффективно обновлены при изменении их входных данных,
и вышеописанную virtual\_dom,
позволяет создавать отзывчивые веб-интерфейсы~\cite{janestreet-opensource}.

\textbf{bonsai}. Расширяет библиотеку incr\_dom, вводя концепцию компонента~\cite{bonsai-history},
представляющих из себя чисто функциональные стейт-машины, из которых можно составлять композицию~\cite{janestreet-bonsai}.

Вообще, использование виртуального DOM нужно так как при взаимодействии с SPA, JavaScript разбирает и выстраивает новый UI
с помощью DOM API, при этом обновлять или изменять новые элементы относительно легко, но процесс вставки новых элементов
идёт крайне медленно~\cite[с.~72]{react-book}. Используя виртуальный DOM можно не взаимодействовать с DOM напрямую.
Преимущество incr\_dom и bonsai в том, что благодаря инкрементальным вычислениям, процесс обнаружения изменений в
виртуальном DOM может иметь быстрее~\cite{minsky-incrdom};
а концепция композиции стейт-машин библиотеки bonsai,хорошо ложиться на
разрабатываемый проект потому что и создаваемый движок является стейт-машиной (согласно пункту~\ref{statemachine_1}).

\subsection{Веб-фреймворки}

Для взаимодействия сервера и клиента требуется обмен сообщений в реальном времени, что лучше всего реализовать
с помощью WebSocket~\cite{mdn-websocket}. Можно рассмотреть библиотеку async\_rpc\_websocket,
которая позволяет описывать протоков удалённого вызова процедур, однако
для своей работы она использует библиотеку Async, что нежелательно.

Поэтому можно использовать веб-фреймворк, поддерживающий технологию WebSocket и реализовать обмен сообщений вручную.
Самым современным таким фреймворком является dream, который отличается идеоматичным для функциональных языков API:
HTTP-обработчики представлены функциями, которые можно композировать в последовательном стиле (умножение),
или в альтернативном (сложение), что можно интерпретировать как алгебраическое кольцо~\cite{dream}.

\subsection{Библиотеки для тестирования}

\textbf{ppx\_inline\_test}. Позволяет специальным синтаксисом создавать тесты
для проверки конкретных свойств
прямо в коде создаваемых библиотек~\cite{rwo-testing}.

\textbf{ppx\_expect}. В предыдущем абзаце упомянут подход, в основном проверяются некоторые конкретные
свойства в определенном сценарии. Иногда, однако, хочется не проверить то или иное свойство, а зафиксировать
и сделать видимым поведение кода. Expect-тесты позволяют сделать именно это.~\cite{rwo-testing}

\subsection{Библиотеки сериализации и десериализации}

Во многих языках программирования (де)сериализация построена на том, что
информация о типе значения доступна во время выполнения программы и с помощью рефлексии можно
получить например информацию о, например названии поля объекта и его значение и записать их в
сериализуемый результат. В OCaml, напротив, во время исполнения отсутствует всякое представление
о типе, т.е. например значение <<0>> типа <<int>> будет иметь то же самое представление в памяти что и
значение <<None>> типа <<option>> или значение <<false>> типа <<bool>>~\cite{rwo-runtime-memory}.

Поэтому (де)сериализация в OCaml построена на другом принципе~-- должна быть определена функция которая
принимает сериализуемый объект и возвращает сериализованное представление (в случае сериализации),
или принимает сериализованное представление и возвращает десериализованный объект (в случае десериализации).
Такие функции программист может определить вручную, но это утомительно и при каждом структуры типа данных,
требуется изменять код функций. Решить эту проблему помогают препроцессорные расширения для OCaml, позволяющие
на этапе компиляции из определения типа данных сгенерировать функции для десериализации и сериализации.
Основные такие библиотеки, предоставляющие сериализованное представление и препроцессорные расширения,
являются Yojson (для JSON~\cite{rwo-json}) и Sexplib (для S-выражений~\cite{rwo-sexp}).

Использование S-выражений является более традиционным в функциональных языках программирования и их
уже используют выбранные библиотеки Core и Bonsai, поэтому для консистентности принято решение
не использовать JSON в проекте.

\subsection{Среда разработки и система сборки}

После многолетней запутанности в системах сборки OCaml, стала стандартом де-факто система сборки Dune;
подобное можно сказать про пакетный менеджер Opam~\cite{anil-gemma-qcon}.

Поддерживается интегрированная среда разработки Visual Studio Code через плагин OCaml Platform~\cite{rwo-platform}.

\section{CSS-фреймворк}

Для оформления пользовательского интерфейса требуется CSS-фреймворк, обеспечивающий
отзывчивую вёрстку (responsive layout)~-- колонки, контейнеры и т.д.;
готовые компоненты~-- кнопки, таблицы, формы и т.д. Самым известным таким фреймворком является Bootstrap,
однако выбор сделан в пользу Bulma, который отличается оригинальностью и простотой~\cite{bulma-vs-bootstrap}.

\section{Выбор библиотек и подходов}

Окончательный перечень используемых средств разработки представлен в таблице~\ref{usedtoolstable}.

\begin{centering}
      \begin{longtable}{|l|p{8cm}|}
            \caption{Используемые средства разработки} \label{usedtoolstable}                                                                                                                                              \\

            \hline \multicolumn{1}{|c|}{\textbf{Название}}                                                                                                      & \multicolumn{1}{c|}{\textbf{Описание}}                   \\ \hline
            \endfirsthead

            \multicolumn{2}{c}%
            {\hspace{-12.5cm}{Окончание таблицы \thetable} \vspace{1ex}}                                                                                                                                                   \\
            \hline \multicolumn{1}{|c|}{\textbf{Название}}                                                                                                      & \multicolumn{1}{c|}{\textbf{Описание}}                   \\ \hline
            \endhead

            \href{https://ocaml.org/}{OCaml}                                                                                                                    & язык программирования                                    \\ \hline
            \href{http://ocsigen.org/js_of_ocaml/}{Js\_of\_ocaml}                                                                                               & компилятор OCaml в JavaScript                            \\ \hline
            \href{https://ocsigen.org/lwt/}{Lwt}                                                                                                                & библиотека для конкурентного программирования            \\ \hline
            \href{https://opensource.janestreet.com/core/}{Core}                                                                                                & стандартная библиотека                                   \\ \hline
            \href{https://aantron.github.io/dream/}{Dream}                                                                                                      & web-фреймворк                                            \\ \hline
            \href{https://github.com/janestreet/ppx_inline_test}{ppx\_inline\_test}, \href{https://github.com/janestreet/ppx_expect}{ppx\_expect}               & библиотеки юнит-тестирования                             \\ \hline
            \href{https://github.com/janestreet/sexplib}{Sexplib}                                                                                               & библиотека для сериализации и десериализации S-выражений \\ \hline
            \href{https://bulma.io/}{Bulma}                                                                                                                     & CSS-фреймворк                                            \\ \hline
            \href{https://dune.build/}{Dune}, \href{https://opam.ocaml.org/}{opam}                                                                              & система сборки и пакетный менеджер                       \\ \hline
            \href{https://code.visualstudio.com/}{VS Code}, \href{https://marketplace.visualstudio.com/items?itemName=ocamllabs.ocaml-platform}{OCaml~Platform} & среда разработки и плагин для работы с OCaml             \\ \hline
      \end{longtable}
\end{centering}
