\chapter{Программная реализация}

Здесь будет третья глава. \TODO

При разработке широко использовалось модульное программирования:
язык модулей ML (реализованный в используемом OCaml) позволяет отделять интерфейс модуля от его реализации,
параметризировать модули другими модулями (образуя функторы), включать один модуль в другой и т.д. [\TODO].
Например, функторы были использованы для того чтобы реализовать создаваемый движок от конкретной реализации
вещественных чисел: т.е. все модули в программе, использующие вещественные числа, являются функторами, которые принимают
модули с сигнатурой, описывающей вещественные числа. Это позволяет абстрагироваться от конкретной реализации вещественных,
что позволит на деле использовать разные реализации, например модуль для работы с числами двойной точности из стандартной библиотеки,
или самописную реализацию с использованием дробей на основе чисел двойной точности, или реализацию на основе длинной арифметики.
Несмотря на то, что при реализации клиентской части используются числа с двойной точностью,
главным плюсом такого решения остаётся потенциальная независимость от чисел двойной точности,
что может быть полезно, особенно при использовании решателя алгебраических уравнений.

\section{Решатель алгебраических уравнений}

\TODO

\section{Движок}\label{engine}

\TODO

\subsection{Символьные вычисления}\label{expr}

\TODO

\section{Клиентская часть, одиночный режим}

\TODO

\section{Серверная часть}

\subsection{Получение отличий модели}\label{model-diff-implementation}

\TODO

\section{Клиентская часть, многопользовательский режим}

\TODO
