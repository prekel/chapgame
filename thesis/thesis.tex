\documentclass[oneside,draft,14pt]{extreport}

\usepackage[T2A]{fontenc}
\usepackage[utf8]{inputenc}
\usepackage[russian]{babel}


\usepackage{vmargin}
\setpapersize{A4}
\setmarginsrb{30mm}{20mm}{10mm}{20mm}{0pt}{0mm}{0pt}{13mm}
\usepackage{indentfirst}
\sloppy
\linespread{1.3}
\setlength{\parindent}{1.25cm}
\setlength{\parskip}{1em}


\newcommand\Vvedenie[1]{
\section*{\begin{normalsize}  \begin{center} \MakeUppercase{ #1 } \end{center} \end{normalsize}
}
\addcontentsline{toc}{section}{%
#1}
}

\newcommand\Section[1]{
\refstepcounter{section}
\section*{\begin{normalsize}\raggedright
\arabic{section} #1\end{normalsize}}
\addcontentsline{toc}{section}{%
\arabic{section} #1}
}

\newcommand\Subsection[1]{
\refstepcounter{subsection}
\subsection*{\begin{normalsize}\raggedright
\arabic{section}.\arabic{subsection} #1
\end{normalsize}}\addcontentsline{toc}{subsection}{%
\arabic{section}.\arabic{subsection} #1}}

\usepackage{titlesec}
\titleformat
{\chapter} % command
[display] % shape
{\bfseries\normalsize} % format
{} % label
{} % sep
{
    \vspace{1ex}
    \centering \MakeUppercase
} % before-code
[
] % after-code

\begin{document}

\renewcommand\contentsname{Содержание}
\renewcommand\figurename{Рисунок}
\title{Practical Typesetting}
\author{Name\\ Work}
\date{December 2005}
\maketitle

\setcounter{page}{2}
\tableofcontents
\pagebreak

\Vvedenie{Введение}
ывафыа


\Section{Предисловие}
Этот текст будет на русском языке. Это демонстрация того, что символы кириллицы
в сгенерированном документе (Compile to PDF) отображаются правильно.
Для этого Вы должны установить нужный  язык (russian)
и необходимую кодировку шрифта (T2A).

\Section{Математические формулы}
Кириллические символы также могут быть использованы в математическом режиме.

\Subsection{Формула 1}
\[\frac{1}{2}\]

\end{document}
