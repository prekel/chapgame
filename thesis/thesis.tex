\documentclass[oneside,draft,14pt]{extreport}

\usepackage{fontspec}
\usepackage{xunicode}
\usepackage{xltxtra}

% \usepackage{pdfpages}
% \usepackage{listings}

\usepackage{geometry}
\geometry{left=3cm}
\geometry{right=1cm}
\geometry{top=2cm}
\geometry{bottom=2cm}

\usepackage{indentfirst}
\usepackage{parskip}
\sloppy
\linespread{1.3}
\setlength{\parindent}{1.25cm}
\setlength{\parskip}{1em}

\usepackage{polyglossia}
\setmainlanguage{russian}
\setkeys{russian}{babelshorthands=true}
\setmainfont{Times New Roman}
\setromanfont{Times New Roman} 
\setsansfont{Arial} 
\setmonofont{Courier New} 
\newfontfamily{\cyrillicfont}{Times New Roman} 
\newfontfamily{\cyrillicfontrm}{Times New Roman}
\newfontfamily{\cyrillicfonttt}{Courier New}
\newfontfamily{\cyrillicfontsf}{Arial}
\makeatletter
\addto\captionsrussian@modern{
    \renewcommand{\contentsname}{Содержание}
    \renewcommand{\bibname}{Список использованых источников}
}
\makeatother

% \usepackage{mathtools,unicode-math}
% \setmathfont{XITS Math}
% \numberwithin{equation}{section}

\usepackage{enumerate}   % Тонкая настройка списков
\usepackage{indentfirst} % Красная строка после заголовка
\usepackage{float}       % Расширенное управление плавающими объектами
\usepackage{multirow}    % Сложные таблицы

\usepackage{cite}

% Оформление библиографии и подрисуночных записей через точку
\makeatletter
\renewcommand*{\@biblabel}[1]{\hfill#1.}
\renewcommand*\l@section{\@dottedtocline{1}{1em}{1em}}
\renewcommand{\thefigure}{\thesection.\arabic{figure}} % Формат рисунка секция.номер
\renewcommand{\thetable}{\thesection.\arabic{table}}   % Формат таблицы секция.номер
\def\redeflsection{\def\l@section{\@dottedtocline{1}{0em}{10em}}}
\makeatother

% Библиография: отступы и межстрочный интервал
\makeatletter
\renewenvironment{thebibliography}[1]
    {\Pseudochapter{\bibname}
        \list{\@biblabel{\@arabic\c@enumiv}}
           {\settowidth\labelwidth{\@biblabel{#1}}
            \leftmargin\labelsep
            \itemindent 18.7mm
            \@openbib@code
            \usecounter{enumiv}
            \let\p@enumiv\@empty
            \renewcommand\theenumiv{\@arabic\c@enumiv}
        }
        \setlength{\itemsep}{0pt}
    }
\makeatother


\newcommand\PseudochapterNoTCO[1]{ 
    \begin{normalsize} 
        \bfseries 
        \begin{center} 
            \MakeUppercase{ #1 } 
        \end{center} 
    \par
    \end{normalsize}
}

\newcommand\Pseudochapter[1]{
    \begin{normalsize} 
        \bfseries 
        \begin{center} 
            \MakeUppercase{ #1 } 
        \end{center}
    \end{normalsize}
    \par
    \addcontentsline{toc}{section}{#1}
}

\newcommand\Section[1]{
    \refstepcounter{section}
    \par{
        \bfseries
        \begin{normalsize}
            \raggedright
            \arabic{section} #1
        \end{normalsize}
    }
    \par
    \addcontentsline{toc}{section}{\arabic{section} #1}
}

\newcommand\Subsection[1]{
    \refstepcounter{subsection}
    \par{
        \bfseries
        \begin{normalsize}
            \raggedright
            \arabic{section}.\arabic{subsection} #1
        \end{normalsize}
    }
    \par
    \addcontentsline{toc}{subsection}{\arabic{section}.\arabic{subsection} #1}
}

\usepackage{titlesec}

\begin{document}

\titleformat
{\chapter} % command
[display] % shape
{\bfseries\normalsize} % format
{} % label
{} % sep
{\centering \MakeUppercase} % before-code
[] % after-code
\titlespacing*{\chapter}{0pt}{-40pt}{0pt}

\title{thesis}
\author{Владислав Прекель}
\date{Май 2022}
\maketitle
\pagebreak
\setcounter{page}{2}

\PseudochapterNoTCO{Реферат}
Здесь реферат
\newpage

\tableofcontents
\newpage

\Pseudochapter{Введение}
Это введение представляет собой простой пример применения паракоманды \titleformat.
Почему? Да потому что в прошлом мы просто применяли парамочему не просто?
\newpage

\Section{Предисловие}
Этот текст будет на русском языке. Это демонстрация того, что символы кириллицы
в сгенерированном документе (Compile to PDF) отображаются правильно.
Для этого Вы должны установить нужный  язык (russian)
и необходимую кодировку шрифта (T2A).

\Section{Математические формулы}
Кириллические символы также могут быть использованы в математическом режиме.

\Subsection{Формула 1}
Привет Мир!~\cite{qwf2}

\Section{Текст}
Фваыфвфы~\cite{qwf1}.

\clearpage
\begin{thebibliography}{00}
    \bibitem{qwf1} qwfqwf1 qwfqwf1 qwfqwf1 qwfqwf1 qwfqwf1
    qwfqwf1 qwfqwf1 qwfqwf1 qwfqwf1 qwfqwf1 qwfqwf1 qwfqwf1 qwfqwf1 qwfqwf1 qwfqwf1 qwfqwf1 qwfqwf1 qwfqwf1 qwfqwf1 qwfqwf1 qwfqwf1 qwfqwf1 qwfqwf1

    \bibitem{qwf2} qwfqwf2qwf qwf2qwfqwf2qwfqwf2 qwfqwf2qwfqwf2qwfqw
    f2qwfqwf2qwfqwf2qwf qwf2qwfqwf2qwfqwf2qwfqwf2qwf qwf2qwfqwf2qwfqwf2

    \bibitem{qwf3} qwfqwf2qwfqwf 2qwfqwf2qwfqw f2qwfqwf 2qwfqwf2qwfqw
    f2qwfqwf2qwf qwf2qwfq wf2qwfqwf2qwf qwf2qwfqwf2qwfqwf2 qwfqwf2qwfqwf2

\end{thebibliography}
\par

\end{document}
