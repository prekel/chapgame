\documentclass[oneside,draft,14pt]{extreport}

\usepackage[T2A]{fontenc}
\usepackage[utf8]{inputenc}
\usepackage[russian]{babel}

\usepackage{amssymb}

\usepackage{vmargin}
\setpapersize{A4}
\setmarginsrb{30mm}{20mm}{10mm}{20mm}{0pt}{0mm}{0pt}{0mm}

\usepackage{indentfirst}
\usepackage{parskip}
\sloppy
\linespread{1.3}
\setlength{\parindent}{1.25cm}
\setlength{\parskip}{1em}

\usepackage{cite}

\newcommand\PseudochapterNoTCO[1]{ 
    \begin{normalsize} 
        \bfseries 
        \begin{center} 
            \MakeUppercase{ #1 } 
        \end{center} 
    \par
    \end{normalsize}
}

\newcommand\Pseudochapter[1]{
    \begin{normalsize} 
        \bfseries 
        \begin{center} 
            \MakeUppercase{ #1 } 
        \end{center}
    \end{normalsize}
    \par
    \addcontentsline{toc}{section}{#1}
}

\newcommand\Section[1]{
    \refstepcounter{section}
    \par{
        \bfseries
        \begin{normalsize}
            \raggedright
            \arabic{section} #1
        \end{normalsize}
    }
    \par
    \addcontentsline{toc}{section}{\arabic{section} #1}
}

\newcommand\Subsection[1]{
    \refstepcounter{subsection}
    \par{
        \bfseries
        \begin{normalsize}
            \raggedright
            \arabic{section}.\arabic{subsection} #1
        \end{normalsize}
    }
    \par
    \addcontentsline{toc}{subsection}{\arabic{section}.\arabic{subsection} #1}
}

\makeatletter
\renewcommand*{\@biblabel}[1]{\hfill#1.}
\makeatother

\usepackage{titlesec}

\begin{document}

\titleformat
{\chapter} % command
[display] % shape
{\bfseries\normalsize} % format
{} % label
{} % sep
{\centering \MakeUppercase} % before-code
[] % after-code

\titlespacing*{\chapter}{0pt}{-40pt}{10pt}


\renewcommand\contentsname{Содержание}
\renewcommand\figurename{Рисунок}
\renewcommand\bibname{Список использованых источников}
\title{Practical Typesetting}
\author{Name\\ Work}
\date{December 2005}
\maketitle
\pagebreak

\setcounter{page}{2}

\PseudochapterNoTCO{Реферат}
Здесь реферат
\newpage

\tableofcontents
\newpage

\Pseudochapter{Введение}
Это введение представляет собой простой пример применения паракоманды \titleformat.
Почему? Да потому что в прошлом мы просто применяли парамочему не просто?
\newpage


\Section{Предисловие}
Этот текст будет на русском языке. Это демонстрация того, что символы кириллицы
в сгенерированном документе (Compile to PDF) отображаются правильно.
Для этого Вы должны установить нужный  язык (russian)
и необходимую кодировку шрифта (T2A).

\Section{Математические формулы}
Кириллические символы также могут быть использованы в математическом режиме.

\Subsection{Формула 1}
\[\frac{1}{2}\]
\[\frac{1}{3} \leqslant \frac{0}{0}\]

\Section{Текст}
Blablabla said Nobody ~\cite{Nobody06}.

\addcontentsline{toc}{section}{\bibname}
\bibliography{thesis}{}
\bibliographystyle{gost780s}

\end{document}
