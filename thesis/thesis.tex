%!TEX TS-program = xelatex

% Author: Amet Umerov (admin@amet13.name)
% https://github.com/Amet13/master-thesis

\documentclass[xetex,a4paper,14pt]{extarticle} % 14й шрифт
%%% Преамбула %%%
% Много позаимствовано из https://github.com/Amet13/master-thesis/blob/619de30308055fb46d432c4d03a58eeff43e6198/preamble.tex

\documentclass[xetex,a4paper,14pt]{extreport}

\usepackage{fontspec} % XeTeX
\usepackage{xunicode} % Unicode для XeTeX
\usepackage{xltxtra}  % Верхние и нижние индексы
\usepackage{pdfpages} % Вставка PDF

% \usepackage{listings} % Оформление исходного кода
% \lstset{
%     basicstyle=\small\ttfamily, % Размер и тип шрифта
%     breaklines=true,            % Перенос строк
%     tabsize=2,                  % Размер табуляции
%     frame=single,               % Рамка
%     literate={--}{{-{}-}}2,     % Корректно отображать двойной дефис
%     literate={---}{{-{}-{}-}}3  % Корректно отображать тройной дефис
% }

% Шрифты
\defaultfontfeatures{Ligatures=TeX}
\setmainfont{Times New Roman}
\newfontfamily\cyrillicfont{Times New Roman}
%\setmonofont{FreeMono} % Моноширинный шрифт для оформления кода

\usepackage{indentfirst}
% \usepackage{parskip}
\sloppy
\linespread{1.3}
\setlength{\parindent}{1.25cm}
% \setlength{\parskip}{0pt}

% Формулы
\usepackage[fleqn]{mathtools}
\usepackage{unicode-math}% Не совместим с amsmath
% \setmathfot{XITS Math}             % Шрифт для формул: https://github.com/khaledhosny/xits-math
% \numberwithin{equation}{section}    % Формула вида секция.номер
\setlength\mathindent{1.25cm}
\usepackage{chngcntr}
\counterwithout{equation}{section}

% Русский язык
\usepackage{polyglossia}
\setdefaultlanguage{russian}
\makeatletter
\addto\captionsrussian@modern{
    \renewcommand{\contentsname}{Содержание}
}
\makeatother

% Абзацы и списки
\usepackage{enumerate}   % Тонкая настройка списков
% \usepackage{indentfirst} % Красная строка после заголовка
\usepackage{float}       % Расширенное управление плавающими объектами
\usepackage{multirow}    % Сложные таблицы

% Пути к каталогам с изображениями
\usepackage{graphicx} % Вставка картинок и дополнений
\graphicspath{{images/}}

% Формат подрисуночных записей
\usepackage{chngcntr}

% Сбрасываем счетчик таблиц и рисунков в каждой новой главе
% \counterwithin{figure}{section}
% \counterwithin{table}{section}

\newcommand\Author{Владислав Прекель}
\newcommand\Topic{Система физического моделирования на основе априорного подхода обнаружения столкновений}

% Гиперссылки
\usepackage{hyperref}
\hypersetup{
    colorlinks, urlcolor={black}, % Все ссылки черного цвета, кликабельные
    linkcolor={black}, citecolor={black}, filecolor={black},
    pdfauthor={\Author},
    pdftitle={\Topic}
}

% Оформление библиографии и подрисуночных записей через точку
\makeatletter
\renewcommand*{\@biblabel}[1]{\hfill#1.}
\renewcommand*\l@section{\@dottedtocline{1}{1em}{1em}}
\renewcommand{\thefigure}{\thesection.\arabic{figure}} % Формат рисунка секция.номер
\renewcommand{\thetable}{\arabic{table}}   % Формат таблицы секция.номер
\def\redeflsection{\def\l@section{\@dottedtocline{1}{0em}{10em}}}
\makeatother

% \renewcommand{\baselinestretch}{1.3} % Полуторный межстрочный интервал
% \parindent 1.25cm % Абзацный отступ

\sloppy             % Избавляемся от переполнений
\hyphenpenalty=1000 % Частота переносов
\clubpenalty=10000  % Запрещаем разрыв страницы после первой строки абзаца
\widowpenalty=10000 % Запрещаем разрыв страницы после последней строки абзаца

% Отступы у страниц
\usepackage{geometry}
\geometry{left=3cm}
\geometry{right=1cm}
\geometry{top=2cm}
\geometry{bottom=2cm}

% Списки
\usepackage{enumitem}
\setlist[enumerate,itemize]{leftmargin=0pt,itemindent=1.70cm,labelsep=1ex,topsep=0em}
\setlist{nolistsep}                          % Нет отступов между пунктами списка
\renewcommand{\labelitemi}{-}                % Маркер списка -
\renewcommand{\labelenumi}{\asbuk{enumi}}    % Список второго уровня
\renewcommand{\labelenumii}{\arabic{enumii}} % Список третьего уровня

% Содержание
\usepackage{tocloft}
\renewcommand{\cfttoctitlefont}{\vspace{-2.5cm}\hspace{0.38\textwidth}\bfseries\MakeTextUppercase} % СОДЕРЖАНИЕ
\renewcommand{\cftsecfont}{\hspace{0pt}}            % Имена секций в содержании не жирным шрифтом
\renewcommand\cftsecleader{\cftdotfill{\cftdotsep}} % Точки для секций в содержании
\renewcommand\cftsecpagefont{\mdseries}
\renewcommand{\cftchapfont}{\hspace{0pt}}            % Имена секций в содержании не жирным шрифтом
\renewcommand\cftchapleader{\cftdotfill{\cftdotsep}} % Точки для секций в содержании
\renewcommand\cftchappagefont{\mdseries}
\setlength{\cftbeforechapskip}{0em}            % Номера страниц не жирные
\setlength{\cftbeforesecskip}{0em}
\setlength{\cftbeforesubsecskip}{0em}
% \setlength{\cftbeforesubsubsecskip}{0em}
\setcounter{tocdepth}{3}                            % Глубина оглавления, до subsubsection


% Нумерация страниц посередине сверху
% \usepackage{fancyhdr}
% \pagestyle{fancy}
% \fancyhf{}
% \cfoot{\textrm{\thepage}}
% \fancyheadoffset{0mm}
% \fancyfootoffset{0mm}
% \setlength{\headheight}{17pt}
% \renewcommand{\headrulewidth}{0pt}
% \renewcommand{\footrulewidth}{0pt}
% \fancypagestyle{plain}{
%     \fancyhf{}
%     \cfoot{\textrm{\thepage}}
% }

% Формат подрисуночных надписей
\RequirePackage{caption}
\DeclareCaptionLabelSeparator{defffis}{ -- } % Разделитель
\captionsetup[figure]{justification=centering, labelsep=defffis, format=plain} % Подпись рисунка по центру
\captionsetup[table]{justification=raggedright, labelsep=defffis, format=plain, singlelinecheck=false} % Подпись таблицы слева
\addto\captionsrussian{\renewcommand{\figurename}{Рисунок}} % Имя фигуры

% Пользовательские функции
% \newcommand{\addimg}[4]{ % Добавление одного рисунка
%     \begin{figure}
%         \centering
%         \includegraphics[width=#2\linewidth]{#1}
%         \caption{#3} \label{#4}
%     \end{figure}
% }
% \newcommand{\addimghere}[4]{ % Добавить рисунок непосредственно в это место
%     \begin{figure}[H]
%         \centering
%         \includegraphics[width=#2\linewidth]{#1}
%         \caption{#3} \label{#4}
%     \end{figure}
% }
% \newcommand{\addtwoimghere}[5]{ % Вставка двух рисунков
%     \begin{figure}[H]
%         \centering
%         \includegraphics[width=#2\linewidth]{#1}
%         \hfill
%         \includegraphics[width=#3\linewidth]{#2}
%         \caption{#4} \label{#5}
%     \end{figure}
% }

% Заголовки секций в оглавлении в верхнем регистре
\usepackage{textcase}
% \makeatletter
% \let\oldcontentsline\contentsline
% \def\contentsline#1#2{
%     \expandafter\ifx\csname l@#1\endcsname\l@section
%         \expandafter\@firstoftwo
%     \else
%         \expandafter\@secondoftwo
%     \fi
%     {\oldcontentsline{#1}{\MakeTextUppercase{#2}}}
%     {\oldcontentsline{#1}{#2}}
% }
% \makeatother

% Оформление заголовков
\usepackage[compact,explicit]{titlesec}

\titleformat{\chapter}
{}
{\bfseries\thechapter~#1}
{0pt}
{}
[]
\titlespacing*{\chapter}
{\parindent}
{-2em}
{1em}

\titleformat{\section}
{}
{\bfseries\thesection~#1}
{0pt}
{}
[]
\titlespacing*{\section}
{\parindent}
{1em}
{1em}

\titleformat{\subsection}
{}
{\bfseries\thesubsection~#1}
{0pt}
{}
[]
\titlespacing*{\subsection}
{\parindent}
{1em}
{1em}

% \titleformat{\subsection}
% [block]{\vspace{1em}}{}{12.5mm}{
%     \bfseries\thesubsection~#1
% }
% \titleformat{\subsubsection}
% [block]{\vspace{1em}\normalsize}{}{12.5mm}{
%     \bfseries\thesubsubsection~#1
%     \vspace{1em}
% }
% \titleformat{\paragraph}
% [block]{\vspace{1em}}{}{12.5mm}{
%     \bfseries\MakeTextUppercase{#1}
% }

% Секции без номеров (введение, заключение...), вместо section*{}
\newcommand{\AnonchapterNoToc}[1]{
    \clearpage
    \phantomsection % Корректный переход по ссылкам в содержании
    \begin{center}
        \bfseries\MakeTextUppercase{#1}
    \end{center}
    \par
}
% Секции без номеров (введение, заключение...), вместо section*{}
\newcommand{\Anonchapter}[1]{
    \AnonchapterNoToc{#1}
    \addcontentsline{toc}{chapter}{#1}
}

% % Секция для аннотации (она не включается в содержание)
% \newcommand{\annotation}[1]{
%     \paragraph{\centerline{{#1}}\vspace{1em}}
% }

% Секции для приложений
% \newcommand{\appsection}[1]{
%     \phantomsection
%     \paragraph{\centerline{{#1}}}
%     \addcontentsline{toc}{section}{{#1}}
% }

% Библиография: отступы и межстрочный интервал
\makeatletter
\renewenvironment{thebibliography}[1]
{\section*{\refname}
    \list{\@biblabel{\@arabic\c@enumiv}}
    {\settowidth\labelwidth{\@biblabel{#1}}
        \leftmargin\labelsep
        \itemindent 16.7mm
        \@openbib@code
        \usecounter{enumiv}
        \let\p@enumiv\@empty
        \renewcommand\theenumiv{\@arabic\c@enumiv}
    }
    \setlength{\itemsep}{0pt}
}
\makeatother

\usepackage{lastpage} % Подсчет количества страниц
% \setcounter{page}{3}  % Начало нумерации страниц

\usepackage{totcount}
\newtotcounter{citenum}
\def\oldbibitem{} \let\oldbibitem=\bibitem
\def\bibitem{\stepcounter{citenum}\oldbibitem}

\usepackage[figure,table,equation]{totalcount}

\usepackage{setspace}
\usepackage[normalem]{ulem} 

\newenvironment{Underequation}{
    \noindent
    где
    \hspace{-5pt}
    \setlength{\parindent}{4.5ex}
}{
    \setlength{\parindent}{1.25cm}
    \vspace{2mm}
}

\newcommand\TODO{{\color{red}TODO}}

\newcommand\Target{разработка физического движка, использующего априорный подход для обнаружения столкновений}
\newcommand\Tasks{\begin{itemize}
        \item определить модель и теоретизировать математическую базу, требующуюся для решения моделирования;
        \item провести обзор используемых технологий при разработке;
        \item программно реализовать физический движок и интерактивную демонстрацию его работы.
    \end{itemize}}

\begin{document}
\setlength{\abovedisplayskip}{-2mm}
\setlength{\belowdisplayskip}{1em}
 % Подключаем преамбулу

%%% Начало документа
\begin{document}

% 1 страница
% \includepdf{extra/pz} % Пояснительная записка

% 2 страница (не нумеруется, но учитывается)
% \includepdf[pages={1,2}]{extra/task} % Задание на ВКР печатается на одном листе с двух сторон

% Помимо ПЗ и задания, в ВКР также вкладываются:
% * отзыв руководителя (otzyv.pdf), на 1 странице с двух сторон
% * рецензия (review.pdf), на 1 странице с двух сторон
% * отчет по антиплагиату (antiplagiat.pdf)

% 3 страница
% \input{inc/0-annotation} % Аннотация


титульный лист
\clearpage


\setcounter{page}{2}

\AnonsectionNoToc{Реферат}
Это реферат. О ВКР.
Это реферат. О ВКР.
Это реферат. О ВКР.
Это реферат. О ВКР.
Это реферат. О ВКР.
Это реферат. О ВКР.
Это реферат. О ВКР.
Это реферат. О ВКР.
Это реферат. О ВКР.
Это реферат. О ВКР.
Это реферат. О ВКР.
Это реферат. О ВКР.
Это реферат. О ВКР.
Это реферат. О ВКР.
Это реферат. О ВКР.
Это реферат. О ВКР.
Это реферат. О ВКР.
Это реферат. О ВКР.
Это реферат. О ВКР.
Это реферат. О ВКР.
Это реферат. О ВКР.

Это реферат. О ВКР.
Это реферат. О ВКР.
Это реферат. О ВКР.


\clearpage
\tableofcontents % Содержание 

\Anonsection{Введение}

Чтото такое. 
Чтото такое. 
Чтото такое. Чтото такое. Чтото такое. Чтото такое. Чтото такое. 
Чтото такое. Чтото такое. Чтото такое. Чтото такое. Чтото такое. Чтото такое. Чтото такое.
 Чтото такое. Чтото такое. Чтото такое. Чтото такое. Чтото такое. Чтото такое. Чтото такое. Чтото такое. 

Чтото такое. Чтото такое. Чтото такое. Чтото такое. Чтото такое. Чтото такое. Чтото такое. Чтото такое. Чтото такое. Чтото такое. Чтото такое. 

\clearpage        % Введение
\section{Анализ}

\subsection{Механизм}

Анализ. Анализ. Анализ. Анализ. Анализ. Анализ. Анализ. Анализ. Анализ. Анализ. Анализ.
Анализ. Анализ. Анализ. Анализ. Анализ.
Анализ. Анализ. Анализ. Анализ. Анализ. Анализ. Анализ. Анализ. Анализ. Анализ. Анализ.
Анализ. Анализ. Анализ. Анализ. Анализ.
Анализ. Анализ. Анализ. Анализ. Анализ. Анализ. Анализ. Анализ. Анализ. Анализ. Анализ.
Анализ. Анализ. Анализ. Анализ. Анализ.
Анализ. Анализ. Анализ. Анализ. Анализ. Анализ. Анализ. Анализ. Анализ. Анализ. Анализ.
Анализ. Анализ. Анализ. Анализ. Анализ.
Анализ. Анализ. Анализ. Анализ. Анализ. Анализ. Анализ. Анализ. Анализ. Анализ. Анализ.
Анализ. Анализ. Анализ. Анализ. Анализ.
Анализ. Анализ. Анализ. Анализ. Анализ. Анализ. Анализ. Анализ. Анализ. Анализ. Анализ.
Анализ. Анализ. Анализ. Анализ. Анализ.
Анализ. Анализ. Анализ. Анализ. Анализ. Анализ. Анализ. Анализ. Анализ. Анализ. Анализ.
Анализ. Анализ. Анализ. Анализ. Анализ.
Анализ. Анализ. Анализ. Анализ. Анализ. Анализ. Анализ. Анализ. Анализ. Анализ. Анализ.
Анализ. Анализ. Анализ. Анализ.

Анализ. Анализ. Анализ. Анализ. Анализ. Анализ. Анализ. Анализ. Анализ. Анализ. Анализ.
Анализ. Анализ. Анализ. Анализ. Анализ.
Анализ. Анализ. Анализ. Анализ. Анализ. Анализ. Анализ. Анализ. Анализ. Анализ. Анализ.
Анализ. Анализ. Анализ. Анализ. Анализ.
Анализ. Анализ. Анализ. Анализ. Анализ. Анализ. Анализ. Анализ. Анализ. Анализ. Анализ.
Анализ. Анализ. Анализ. Анализ. Анализ.
Анализ. Анализ. Анализ. Анализ. Анализ. Анализ. Анализ. Анализ. Анализ. Анализ. Анализ.
Анализ. Анализ. Анализ. Анализ. Анализ.
Анализ. Анализ. Анализ. Анализ. Анализ. Анализ. Анализ. Анализ. Анализ. Анализ. Анализ.
Анализ. Анализ. Анализ. Анализ. Анализ.
Анализ. Анализ. Анализ. Анализ. Анализ. Анализ. Анализ. Анализ. Анализ. Анализ. Анализ.
Анализ. Анализ. Анализ. Анализ. Анализ.
Анализ. Анализ. Анализ. Анализ. Анализ. Анализ. Анализ. Анализ. Анализ. Анализ. Анализ.
Анализ. Анализ. Анализ. Анализ. Анализ.
Анализ. Анализ. Анализ. Анализ. Анализ. Анализ. Анализ. Анализ. Анализ. Анализ. Анализ.
Анализ. Анализ. Анализ. Анализ. Анализ.
Анализ. Анализ. Анализ. Анализ. Анализ. Анализ. Анализ. Анализ. Анализ. Анализ. Анализ.

Анализ. Анализ. Анализ. Анализ. Анализ.
Анализ. Анализ. Анализ. Анализ. Анализ. Анализ. Анализ. Анализ. Анализ. Анализ. Анализ.
Анализ. Анализ. Анализ. Анализ. Анализ.
Анализ. Анализ. Анализ. Анализ. Анализ. Анализ. Анализ. Анализ. Анализ. Анализ. Анализ.
Анализ. Анализ. Анализ. Анализ. Анализ.
Анализ. Анализ. Анализ. Анализ. Анализ. Анализ. Анализ. Анализ. Анализ. Анализ. Анализ.
Анализ. Анализ. Анализ. Анализ. Анализ.
Анализ. Анализ. Анализ. Анализ. Анализ. Анализ. Анализ. Анализ. Анализ. Анализ. Анализ.
Анализ. Анализ. Анализ. Анализ. Анализ.
Анализ. Анализ. Анализ. Анализ. Анализ. Анализ. Анализ. Анализ. Анализ. Анализ. Анализ.
Анализ. Анализ. Анализ. Анализ. Анализ.
Анализ. Анализ. Анализ. Анализ. Анализ. Анализ. Анализ. Анализ. Анализ. Анализ. Анализ.
Анализ. Анализ. Анализ. Анализ. Анализ.
Анализ. Анализ. Анализ. Анализ. Анализ. Анализ. Анализ. Анализ. Анализ. Анализ. Анализ.
Анализ. Анализ. Анализ. Анализ. Анализ.
Анализ. Анализ. Анализ. Анализ. Анализ. Анализ. Анализ. Анализ. Анализ. Анализ. Анализ.
Анализ. Анализ. Анализ. Анализ. Анализ.
Анализ. Анализ. Анализ. Анализ. Анализ. Анализ. Анализ. Анализ. Анализ. Анализ. Анализ.
Анализ. Анализ. Анализ. Анализ. Анализ.
Анализ. Анализ. Анализ. Анализ. Анализ. Анализ. Анализ. Анализ. Анализ. Анализ. Анализ.
Анализ. Анализ. Анализ. Анализ. Анализ.
Анализ. Анализ. Анализ. Анализ. Анализ. Анализ. Анализ. Анализ. Анализ. Анализ. Анализ.
Анализ. Анализ. Анализ. Анализ. Анализ.

Анализ. Анализ. Анализ. Анализ. Анализ. Анализ. Анализ. Анализ. Анализ. Анализ. Анализ.
Анализ. Анализ. Анализ. Анализ. Анализ.
Анализ. Анализ. Анализ. Анализ. Анализ. Анализ. Анализ. Анализ. Анализ. Анализ. Анализ.
Анализ. Анализ. Анализ. Анализ. Анализ.
Анализ. Анализ. Анализ. Анализ. Анализ. Анализ. Анализ. Анализ. Анализ. Анализ. Анализ.
Анализ. Анализ. Анализ. Анализ. Анализ.
Анализ. Анализ. Анализ. Анализ. Анализ. Анализ. Анализ. Анализ. Анализ. Анализ. Анализ.
Анализ. Анализ. Анализ. Анализ. Анализ.
Анализ. Анализ. Анализ. Анализ. Анализ. Анализ. Анализ. Анализ. Анализ. Анализ. Анализ.
Анализ. Анализ. Анализ. Анализ. Анализ.
Анализ. Анализ. Анализ. Анализ. Анализ. Анализ. Анализ. Анализ. Анализ. Анализ. Анализ.
Анализ. Анализ. Анализ. Анализ. Анализ.
Анализ. Анализ. Анализ. Анализ. Анализ. Анализ. Анализ. Анализ. Анализ. Анализ. Анализ.
Анализ. Анализ. Анализ. Анализ. Анализ.
Анализ. Анализ. Анализ. Анализ. Анализ. Анализ. Анализ. Анализ. Анализ. Анализ. Анализ.
Анализ. Анализ. Анализ. Анализ. Анализ.
Анализ. Анализ. Анализ. Анализ. Анализ. Анализ. Анализ. Анализ. Анализ. Анализ. Анализ.
Анализ. Анализ. Анализ. Анализ. Анализ.
Анализ. Анализ. Анализ. Анализ. Анализ. Анализ. Анализ. Анализ. Анализ. Анализ. Анализ.
Анализ. Анализ. Анализ. Анализ. Анализ.
Анализ. Анализ. Анализ. Анализ. Анализ. Анализ. Анализ. Анализ. Анализ. Анализ. Анализ.
Анализ. Анализ. Анализ. Анализ. Анализ.
Анализ. Анализ. Анализ. Анализ. Анализ. Анализ. Анализ. Анализ. Анализ. Анализ. Анализ.
Анализ. Анализ. Анализ. Анализ. Анализ.
Анализ. Анализ. Анализ. Анализ. Анализ. Анализ. Анализ. Анализ. Анализ. Анализ. Анализ.
Анализ. Анализ. Анализ. Анализ. Анализ.
Анализ. Анализ. Анализ. Анализ. Анализ. Анализ. Анализ. Анализ. Анализ. Анализ. Анализ.
Анализ. Анализ. Анализ. Анализ. Анализ.
Анализ. Анализ. Анализ. Анализ. Анализ. Анализ. Анализ. Анализ. Анализ. Анализ. Анализ.
Анализ. Анализ. Анализ. Анализ. Анализ.
Анализ. Анализ. Анализ. Анализ. Анализ. Анализ. Анализ. Анализ. Анализ. Анализ. Анализ.
Анализ. Анализ. Анализ. Анализ. Анализ.
Анализ. Анализ. Анализ. Анализ. Анализ. Анализ. Анализ. Анализ. Анализ. Анализ. Анализ.
Анализ. Анализ. Анализ. Анализ. Анализ.
Анализ. Анализ. Анализ. Анализ. Анализ. Анализ. Анализ. Анализ. Анализ. Анализ. Анализ.
Анализ. Анализ. Анализ. Анализ. Анализ.
Анализ. Анализ. Анализ. Анализ. Анализ. Анализ. Анализ. Анализ. Анализ. Анализ. Анализ.
Анализ. Анализ. Анализ. Анализ. Анализ.
Анализ. Анализ. Анализ. Анализ. Анализ. Анализ. Анализ. Анализ. Анализ. Анализ. Анализ.
Анализ. Анализ. Анализ. Анализ. Анализ.
Анализ. Анализ. Анализ. Анализ. Анализ. Анализ. Анализ. Анализ. Анализ. Анализ. Анализ.
Анализ. Анализ. Анализ. Анализ. Анализ.
Анализ. Анализ. Анализ. Анализ. Анализ. Анализ. Анализ. Анализ. Анализ. Анализ. Анализ.
Анализ. Анализ. Анализ. Анализ. Анализ.
Анализ. Анализ. Анализ. Анализ. Анализ. Анализ. Анализ. Анализ. Анализ. Анализ. Анализ.
Анализ. Анализ. Анализ. Анализ. Анализ.
Анализ. Анализ. Анализ. Анализ. Анализ. Анализ. Анализ. Анализ. Анализ. Анализ. Анализ.
Анализ. Анализ. Анализ. Анализ. Анализ.
Анализ. Анализ. Анализ. Анализ. Анализ. Анализ. Анализ. Анализ. Анализ. Анализ. Анализ.
Анализ. Анализ. Анализ. Анализ. Анализ.
Анализ. Анализ. Анализ. Анализ. Анализ. Анализ. Анализ. Анализ. Анализ. Анализ. Анализ.
Анализ. Анализ. Анализ. Анализ. Анализ.
Анализ. Анализ. Анализ. Анализ. Анализ. Анализ. Анализ. Анализ. Анализ. Анализ. Анализ.
Анализ. Анализ. Анализ. Анализ. Анализ.

\subsection{Формулы}

В (\ref{equation1}) Анализ.

\begin{equation}\label{equation1}
  y_i = \sum_{i=1}^{n} a_{ik}, k = 1,2,...,n
\end{equation}
где \(x\)~-- это чтото.

В (\ref{equation2}) Анализ2.

\begin{equation}\label{equation2}
  y_i = k = 1,2,...,n
\end{equation}
\begin{tabular}{rl}
  где & \(y\)~-- это чтото; \\
      & \(x\)~-- это чтото.
\end{tabular}

\subsection{Формулы 1}

Скорость при равноускоренном движении (\ref{velocityvec_1}) TODO

\begin{equation}\label{velocityvec_1}
  \vec{v}(t) = \vec{v_0} + \vec{a}t
\end{equation}
\begin{Underequation}
  & \(\vec{v}(t)\)~-- вектор скорости тела в момент времени \(t\); \\
  & \(\vec{v_0}\)~-- вектор начальной скорости тела; \\
  & \(\vec{a}\)~-- вектор ускорения тела; \\
  & \(t\)~-- момент времени.
\end{Underequation}

Причём вектор \(\vec{v}(t)\) должен быть сонаправлен вектору \(\vec{v_0}\), а вектор \(\vec{a}\) противонаправлен.
Для того чтобы выяснить, при каких \(t\) сонаправленность векторов \(\vec{v}(t)\) и \(\vec{v_0}\) в уравнении (\ref{velocityvec_1}) соблюдается,
достаточно увидеть, что длина вектора \(\vec{v_0}\) должна быть больше длине вектора \(\vec{a}t\)
и получить неравенство для \(t\) (\ref{constraint_t_1}).

\[
  \left|\vec{v_0}\right| > \left|\vec{a}t\right|
\]
\[
  \left|\vec{v_0}\right| > \left|\vec{a}\right| t
\]
\[
  \frac{\left|\vec{v_0}\right|}{\left|\vec{a}\right| } > t
\]

\newcommand\Constrainttle{
  t < \frac{\left|\vec{v_0}\right|}{\left|\vec{a}\right|}
}

\newcommand\Constrainttge{
  t \geqslant \frac{\left|\vec{v_0}\right|}{\left|\vec{a}\right|}
}

\begin{equation}\label{constraint_t_1}
  \Constrainttle
\end{equation}

А для остальных \(t\), \(\vec{v}(t)\) следует принять нулю. Тогда получится система (\ref{v_system}).

\begin{equation}\label{v_system}
  \vec{v}(t) =
  \begin{cases}
    \vec{v_0} + \vec{a}t, & 0 \leqslant \Constrainttle, \\
    0,                    & \Constrainttge .
  \end{cases}
\end{equation}

Проекции на ось абцисс (\ref{v_x_1}) и ординат (\ref{v_y_1}):

\begin{equation}\label{v_x_1}
  v_x(t) =
  \begin{cases}
    {v_0}_x - a_x t, & 0 \leqslant \Constrainttle, \\
    0,               & \Constrainttge.
  \end{cases}
\end{equation}
\begin{Underequation}
  & \(v_x(t)\)~-- проекция вектора скорости тела \(\vec{v}(t)\) в момент времени \(t\) на ось \(X\); \\
  & \({v_0}_x\)~-- проекция вектора начальной скорости тела \(\vec{v_0}\) на ось \(X\); \\
  & \(a_x\)~-- проекция вектора ускорения тела \(\vec{a}\) на ось \(X\). \\
\end{Underequation}

\begin{equation}\label{v_y_1}
  v_y(t) =
  \begin{cases}
    {v_0}_y - a_y t, & 0  \leqslant \Constrainttle, \\
    0,               & \Constrainttge.
  \end{cases}
\end{equation}
\begin{Underequation}
  & \(v_y(t)\)~-- проекция вектора скорости тела \(\vec{v}(t)\) в момент времени \(t\) на ось \(Y\); \\
  & \({v_0}_y\)~-- проекция вектора начальной скорости тела \(\vec{v_0}\) на ось \(Y\); \\
  & \(a_y\)~-- проекция вектора ускорения тела \(\vec{a}\) на ось \(Y\). \\
\end{Underequation}

Теперь найдём формулу для траектории движения тела. Формуле, соответвующей (\ref{velocityvec_1}),
только для траектории, соответствует (\ref{traectoryvec_1}):

\begin{equation}\label{traectoryvec_1}
  \vec{r}(t) = \vec{r_0} + \vec{v_0}t + \frac{\vec{a}t^2}{2}
\end{equation}
\begin{Underequation}
  & \(\vec{r}(t)\)~-- радиус-вектор положения тела в момент времени \(t\); \\
  & \(\vec{r_0}\)~-- радиус-вектор начального положения тела.
\end{Underequation}

Исходя из (\ref{v_system}), уравнение для траектории с учётом того, что вектор скорости должен быть
противонаправлен вектору ускорения, будет (\ref{r_system_1}):

\begin{equation}\label{r_system_1}
  \vec{r}(t) = \begin{cases}
    \vec{r_0} + \vec{v_0}t + \frac{\vec{a}t^2}{2}, & 0 \leqslant \Constrainttle, \\
    0,                                             & \Constrainttge .
  \end{cases}
\end{equation}

\section{Чтото}

Чтото.
Чтото.
Чтото.
Чтото.
Чтото.  Механизм. Механизм. Механизм. Механизм. Механизм.
Механизм. Механизм. Механизм. Механизм. Механизм. Механизм. Механизм. Механизм.
Механизм. Механизм. Механизм. Механизм. Механизм. Механизм. Механизм. Механизм.
Механизм. Механизм. Механизм. Механизм. Механизм. Механизм. Механизм. Механизм.
Механизм. Механизм. Механизм. Механизм. Механизм. Механизм. Механизм. Механизм.
Механизм. Механизм. Механизм. Механизм. Механизм. Механизм. Механизм. Механизм.
Механизм. Механизм. Механизм.
Чтото.
Чтото.
Чтото.
Чтото.
Чтото.
Чтото.
Чтото.
Чтото.
Чтото.
Чтото.
Чтото.
Чтото.        % Основная часть

% \input{inc/1-pz}           % Постановка задачи
% \input{inc/2-literature}   % Обзор литературных источников
% \input{inc/3-sysanalyz}    % Системный анализ
% \input{inc/4-varanalyz}    % Вариантный анализ
% \input{inc/5-description}  % Описание системы информационной безопасности облачной среды
% \input{inc/6-experiment}   % Экспериментальные исследования
% \input{inc/7-results}      % Анализ полученных результатов
% \input{inc/0-conclusion}   % Заключение
% \input{inc/0-glossary}     % Перечень сокращений и условных обозначений
\begingroup
\renewcommand{\section}[2]{\Anonsection{Список использованных источников}}
\begin{thebibliography}{00}

\bibitem{qwf}
    Qwf, А.А.
    Qws arstarst sratrst /
    А.А. asrtarst //
    Мир tsratarst. -- 2013. -- №1. -- С. 50-55.

\end{thebibliography}
\endgroup

\clearpage
 % Библиографический список

% Приложения
% \input{inc/a-app} % Программа мониторинга уязвимостей в программном обеспечении

\end{document}
%%% Конец документа
