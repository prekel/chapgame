\AnonchapterNoToc{Реферат}

Выпускная квалификационная работа по теме <<\Topic>> содержит
\pageref{LastPage} страниц текстового документа,
% \totalfigures~иллюстраций,
% \totaltables~таблицы,
\totalequations~формул,
% \TODO приложений,
\total{citenum} использованных источников.

\MakeUppercase{
    физический движок,
    обнаружение столкновений,
    априорный подход,
    численные методы,
    OCaml.
}

\newcommand\Target{разработка физического движка использующего априорный подход для обнаружения столкновений}
\newcommand\Tasks{\begin{itemize}
        \item определить модель и теоретизировать математическую базу, требующуюся для решения моделирования;
        \item провести обзор используемых технологий при разработке;
        \item программно реализовать физический движок и интерактивную демонстрацию его работы.
    \end{itemize}}

Цель работы: \Target.

Задачи:

\Tasks
