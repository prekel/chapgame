\chapter{Теоретическая часть}

\section{Описание модели}

Тело~-- абсолютно твёдрое тело в форме круга равномерной плотности
(центр масс в центре круга) обладающее массой (\(m\)), коэффициентом трения (\(\mu\)), 
радиусом (\(r\)), начальной скоростью (\(\vec{v_0}\)), положением (координаты \(x\) и \(y\) или радиус-вектор \(\vec{r}\)).
На тело действует сила трения (\(F_\text{тр.}\)). \TODO

Точка~-- неподвижная точка в пространстве, определена через координаты.

Линия~-- неподвижная прямая линия в пространстве, может быть ограничена точкой с двух или
одной сторон образуя отрезок или луч соответсвенно.

Сцена~-- множество тел, линий, точек и постоянных (например, ускорение свободного падения).

Обновлённая сцена~-- сцена, в которой обновлены параметры тел, линий, точек или постоянных.

Сцена в момент времени~-- сцена, в которой все тела обновлены так, что новая начальная скорость равна скорости в этот момент времени.

\section{Формулы}

Скорость при равноускоренном движении (\ref{velocityvec_1}) \TODO \cite[с.~96]{rowellherbert}.

\begin{equation}\label{velocityvec_1}
  \vec{v}(t) = \vec{v_0} + \vec{a}t
\end{equation}

\begin{Underequation}
  \(\vec{v}(t)\)~-- вектор скорости тела в момент времени \(t\);

  \(\vec{v_0}\)~-- вектор начальной скорости тела;

  \(\vec{a}\)~-- вектор ускорения тела;

  \(t\)~-- момент времени.
\end{Underequation}

Причём вектор \(\vec{v}(t)\) должен быть сонаправлен вектору \(\vec{v_0}\), а вектор \(\vec{a}\) противонаправлен.
Для того чтобы выяснить, при каких \(t\) сонаправленность векторов \(\vec{v}(t)\) и \(\vec{v_0}\) в уравнении (\ref{velocityvec_1}) соблюдается,
достаточно увидеть, что длина вектора \(\vec{v_0}\) должна быть больше длине вектора \(\vec{a}t\)
и получить неравенство для \(t\) (\ref{constraint_t_1}).

% \[
%   \left|\vec{v_0}\right| > \left|\vec{a}t\right|
% \]
% \[
%   \left|\vec{v_0}\right| > \left|\vec{a}\right| t
% \]
% \[
%   \frac{\left|\vec{v_0}\right|}{\left|\vec{a}\right| } > t
% \]

\newcommand\Constrainttle{
  t < \frac{\left|\vec{v_0}\right|}{\left|\vec{a}\right|}
}

\newcommand\Constrainttge{
  t \geqslant \frac{\left|\vec{v_0}\right|}{\left|\vec{a}\right|}
}

\begin{equation}\label{constraint_t_1}
  \Constrainttle
\end{equation}

А для остальных \(t\), \(\vec{v}(t)\) следует принять нулю. Тогда получится система (\ref{v_system}).

\begin{equation}\label{v_system}
  \vec{v}(t) =
  \begin{cases}
    \vec{v_0} + \vec{a}t, & 0 \leqslant \Constrainttle, \\
    0,                    & \Constrainttge .
  \end{cases}
\end{equation}

Проекции на ось абцисс (\ref{v_x_1}) и ординат (\ref{v_y_1}):

\begin{equation}\label{v_x_1}
  v_x(t) =
  \begin{cases}
    {v_0}_x + a_x t, & 0 \leqslant \Constrainttle, \\
    0,               & \Constrainttge.
  \end{cases}
\end{equation}

\begin{Underequation}
  \(v_x(t)\)~-- проекция вектора скорости тела \(\vec{v}(t)\) в момент времени \(t\) на ось \(X\);

  \({v_0}_x\)~-- проекция вектора начальной скорости тела \(\vec{v_0}\) на ось \(X\);

  \(a_x\)~-- проекция вектора ускорения тела \(\vec{a}\) на ось \(X\).
\end{Underequation}

\begin{equation}\label{v_y_1}
  v_y(t) =
  \begin{cases}
    {v_0}_y + a_y t, & 0  \leqslant \Constrainttle, \\
    0,               & \Constrainttge.
  \end{cases}
\end{equation}

\begin{Underequation}
  \(v_y(t)\)~-- проекция вектора скорости тела \(\vec{v}(t)\) в момент времени \(t\) на ось \(Y\);

  \({v_0}_y\)~-- проекция вектора начальной скорости тела \(\vec{v_0}\) на ось \(Y\);

  \(a_y\)~-- проекция вектора ускорения тела \(\vec{a}\) на ось \(Y\).
\end{Underequation}

Теперь найдём формулу для траектории движения тела. Формуле, соответвующей (\ref{velocityvec_1}),
только для траектории, соответствует (\ref{traectoryvec_1}):

\begin{equation}\label{traectoryvec_1}
  \vec{r}(t) = \vec{r_0} + \vec{v_0}t + \frac{\vec{a}t^2}{2}
\end{equation}

\begin{Underequation}
  \(\vec{r}(t)\)~-- радиус-вектор положения тела в момент времени \(t\);

  \(\vec{r_0}\)~-- радиус-вектор начального положения тела.
\end{Underequation}

Исходя из (\ref{v_system}), уравнение для траектории с учётом того, что вектор скорости должен быть
противонаправлен вектору ускорения, будет (\ref{r_system_1}):

\begin{equation}\label{r_system_1}
  \vec{r}(t) = \begin{cases}
    \vec{r_0} + \vec{v_0}t + \frac{\vec{a}t^2}{2}, & 0 \leqslant \Constrainttle, \\
    \vec{r_0},                                     & \Constrainttge .
  \end{cases}
\end{equation}

Соответствующие проекции на ось абцисс (\ref{r_x_1}) и ординат (\ref{r_y_1}):

\begin{equation}\label{r_x_1}
  x(t) =
  \begin{cases}
    x_0 + {v_0}_x t + \frac{a_x t^2}{2}, & 0 \leqslant \Constrainttle, \\
    x_0,                                 & \Constrainttge.
  \end{cases}
\end{equation}

\begin{Underequation}
  \(x(t)\)~-- координата положения тела \(\vec{r}(t)\) в момент времени \(t\) на ось \(X\);

  \(x_0\)~-- координата начального положения тела \(\vec{v_0}\) на ось \(X\).
\end{Underequation}

\begin{equation}\label{r_y_1}
  r_y(t) =
  \begin{cases}
    y_0 + {v_0}_y t + \frac{a_y t^2}{2}, & 0 \leqslant \Constrainttle, \\
    y_0,                                 & \Constrainttge.
  \end{cases}
\end{equation}

\begin{Underequation}
  \(y(t)\)~-- координата положения тела \(\vec{r}(t)\) в момент времени \(t\) на ось \(Y\);

  \(y_0\)~-- координата начального положения тела \(\vec{v_0}\) на ось \(Y\).
\end{Underequation}

Формулы (\ref{r_x_1}) и (\ref{r_y_1}) являются ключевыми в этой работе.
