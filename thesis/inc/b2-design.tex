\chapter{Проектирование}

Кроме создания физического движка, важно продемонстрировать его работу.
Желательно в интерактивном режиме. В XXI веке сложилась ситуация, что веб-приложения
обладают преимуществом, таким что для того чтобы им воспользовался пользователь, достаточно
браузера и просто перейти по ссылке.
Поэтому, решено сделать Single Page Application
(т.е. веб-приложение, которое загружает только один веб-документ,
а затем обновляет содержимое тела этого документа с помощью JavaScript API,
когда необходимо показать другое содержимое \cite{mdn-spa}),
которое в интерактивном режиме отображает состояние модели (описанной в пункте \ref{model_1_3}),
меняющееся с течением времени.

Так как создаваемый движок будет способен работать в реальном времени, существует возможность
создать многопользовательский режим, для которого потребуется серверная часть,
на которую перенесётся работа движка, а клиенты будут получать лишь изменения в модели (подробнее в \ref{model-diff-implementation}).
Соответственно, браузерное интерактивное SPA приложение с графическим интерфейсом назовём клиентской часть.

Так как мы хотим, чтобы движок работал и на клиентской части, и на серверной, следует выбрать язык программирования,
который позволяет создавать и браузерные приложения, и нативные.

При этом для того чтобы сделать веб-приложение интерактивным, браузеры могут работать лишь с JavaScript
с возможность подключения WebAssembly-модулей. WebAssembly~-- относительно новый способ создавать
приложения на языке, отличным от JavaScript; он представляет из себя байткод стековой машины,
который может быть получен из множества языков \cite{wasm}, например
C/С++ \cite{emscripten-about}, Rust \cite{rust-wasm}, C\# \cite{blazor-ru}, F\# \cite{fsbolero} и т.д. \cite{wasm-iwantto}.

Однако существует и другой способ
писать на языке отличном от JavaScript~-- это использования компилятора, который исходный код на нужном языке
компилирует в код на языке JavaScript, пригодный для исполнения в браузере. Поэтому существуют языки программирования,
пригодные для исполнения и в браузере, и на сервере, что приводит к переиспользованию кода и уменьшении времени разработки.
Примеры таких языков: OCaml (подробнее в пункте \ref{jsoo}), F\# \cite[с.~48]{dsyme-hopl}, TypeScript и т.д. \cite{typescript-mayorov}.

При выборе языка программирования важно руководствоваться не только его кроссплатформенностью
(в данном случае кроссплатформенностью можно назвать возможность работать и в браузере, и на сервере),
но ещё и опытом работы с ним; стоит обратить внимание на выразительность, способность к обнаружению ошибок,
производительность, экосистему. В качестве такого языка программирования выбран OCaml.
OCaml является функциональным языком программирования, что можно рассматривать как преимущество
при использовании как инструмента для решения задач численного моделирования \cite{shutov-haskell}.

\section{Язык программирования OCaml}

OCaml (до 2011 - Objective Caml \cite{camlhistory})~-- промышленный язык программирования общего назначения,
в котором особое внимание уделяется выразительности и надёжности \cite{ocamlorg}. Обладает
мультипарадигменностью, но в первую очередь преподноситься как функциональный язык программирования.

Ключевые достоинства и черты OCaml, согласно \cite[c.~3]{yaron2011} и \cite{rwo-prologue}:

\begin{itemize}
      \item \textbf{выразительность}. на OCaml можно создавать более компактные, простые и понятные системы,
            чем на таких языках, как Java или C\#;
      \item \textbf{обнаружение ошибок}. Cуществует удивительно
            широкий круг ошибок, против которых система типов эффективна, включая многие ошибки, которые довольно трудно
            обнаружить с помощью тестирования;
      \item \textbf{производительность}. Производительность OCaml находится на одном уровне или лучше,
            чем у Java, и на расстоянии вытянутой руки от таких языков, как C или C++. В добавок к высококачественной генерации нативного кода,
            OCaml имеет инкрементальный сборщик мусора, который может быть настроен на выполнение небольших
            фрагментов работы за раз, что делает его более подходящим для приложений мягкого реального времени,
            таких как трейдинг;
      \item сборка мусора для автоматического управления памятью, которая сейчас является общей чертой современных языков высокого уровня;
      \item функции первого порядка, которые можно передавать как обычные значения, как в JavaScript, Common Lisp и C\#;
      \item статическая типизация увеличивает производительность и уменьшает число ошибок во время исполнения, как в Java или C\#;
      \item параметрический полиморфизм, позволяющий создавать абстракции, работающие с различными типами данных,
            подобно джереникам из Java, Rust, C\# или шаблонам из C++;
      \item хорошая поддержка иммутабельного программирования, т.е. программирования без деструктивных обновлений в структурах данных.
            Такой подход представлен в традиционных функциональных языках программирования, таких как Scheme, а так же часто встречается
            во всём от распределенных фреймворков для работы с большими данными до UI-тулкитов;
      \item вывод типов, который позволяет не указывать тип каждой переменной в программе;
            Вместо этого, типы выводятся из того, как используется значение. В мейнстримных языках
            такое есть на уровне локальных переменных, например ключевое слово <<var>> в C\# или ключевое слово <<auto>> в C++.
      \item алгебраические типы данных и паттерн-матчинг позволяют определять и манипулировать сложными структурами данных;
            Так же доступны в Scala, Rust, F\#.

\end{itemize}

\section{Обзор экосистемы языка OCaml}

\TODO

\subsection{Компиляторы OCaml в JavaScript}\label{jsoo}

\textbf{ReScript}.

\textbf{Js\_of\_ocaml}.

Прочее. Ранее существовал компилятор ocamljs, который для генерации JavaScript
использовал внутреннее <<lambda-представление>> компилятора OCaml \cite{ocamljs-lambda}.
Так как такое представление является нестабильным (т.е. меняется от релиза к релизу языка OCaml) \cite{rwo-backend},
такой подход серьёзно усложняет поддержку последних версий языка. В отличие от подхода js\_of\_ocaml,
который использует более стабильный байткод, который кроме того, 
помогает добиться большей производительности \cite[с.~13]{vouillon-jsoo}. 
Ещё один подход~-- использовать WebAssembly при работе с OCaml, однако 
инструментарий для такого подхода сейчас находиться в зачаточном состоянии \TODO.

\subsection{Стандартные библиотеки}

\TODO

ocaml: stdlib, base, core, containers, batteries,

rescipt: belt, tablecloth

\subsection{ML-синтаксис и Reason-синтаксис}

\TODO

\subsection{Библиотеки конкурентного программирования}

В своей книге \cite{rwo-ru} Я. Мински, А. Мадхавапедди, Дж. Хикки отметили, что:
<<Логика работы программ, взаимодействующих с внешним миром, часто предполагает
ожидание: ожидание щелчка мышью, ожидание завершения операции чтения
данных с диска или ожидание освобождения места в выходном сетевом буфере.
Даже в не самых сложных интерактивных приложениях с успехом можно использовать
приемы конкурентного программирования, например для ожидания наступления
нескольких событий или немедленной реакции на первое наступившее событие.
Один из подходов к организации конкурентного выполнения заключается
в использовании системных потоков выполнения. Данный подход является
доминирующим в таких языках программирования, как Java или C\#. В этой модели
для каждой задачи, которой может потребоваться приостановиться в ожидании
некоторого события, выделяется отдельный поток выполнения, который можно
заблокировать, не останавливая работу самой программы.
Другой подход применяется в однопоточных программах и заключается в
использовании цикла событий, в рамках которого реализуется реакция на внешние
события, такие как тайм-ауты или щелчки мышью, путем вызова функций,
специально зарегистрированных для этого. Этот подход часто используется в языках
программирования, подобных языку JavaScript, имеющему однопоточную среду
выполнения, а также во многих библиотеках поддержки графического
интерфейса пользователя.
Каждый из данных механизмов имеет собственные достоинства и недостатки.
Система потоков требует значительного объема памяти и других системных
ресурсов. Кроме того, операционная система может произвольно прерывать
выполнение одного потока, чтобы передать управление другому, что требует
от программиста проявлять особую осторожность и защищать совместно используемые
ресурсы с применением блокировок и условных переменных, вследствие чего
легко допустить ошибку.
Однопоточные системы, управляемые событиями, с другой стороны, в каждый
момент времени решают только одну задачу и не требуют применения сложных
механизмов синхронизации. Однако перевернутая организация программ,
управляемых событиями, часто приводит к тому, что вам нередко приходится прибегать
к весьма неуклюжим уловкам, чтобы обеспечить некоторое подобие
конкурентного выполнения в цикле событий, что может завести в труднопроходимый
лабиринт разнообразных ситуаций обработки событий.>>

При этом асинхронность в JavaScript работает так же через цикл событий (event loop).
Это тоже роднит семантику OCaml с семантикой JavaScript, как указано в пункте \ref{jsoo}.

Основные библиотеки~-- Lwt и Async. Lwt (<<lightweight threads>>~-- легковесные потоки)
позволяет писать программы с участием потоков в монадическом стиле, что позволяет писать
асинхронный код как обычный ML-код \cite[с.~1]{vouillon-lwt}. Async создавалась с оглядкой на Lwt,
но компании Jane Street, которая создала Async, требовалась другая обработка ошибок,
более лучший контроль над параллелизмом \cite{announcing-async}. Но при этом, эти две библиотеки
не совместимы между собой, т.е. один проект не может полноценно использовать обе библиотеки
вместе полноценно \cite{rgrinberg-async}. Из-за этого в экосистеме существует деление среди библиотек,
использующих асинхронность~-- часть библиотек используют Lwt, часть Async. Получается, следует
учитывать что одни библиотеки и фреймворки могут быть не совместимы с другими.
% При этом большинство
% библиотек, использующий Async, написаны так же Jane Street и при создании проекта приходится выбирать,
% использовать стек Jane Street, в котором 

\TODO

\subsection{Библиотеки для построения пользовательского интерфейса}

\TODO

js\_of\_ocaml: jsoo-react, ocaml-vdom, virtual\_dom, incr\_dom, bonsai

rescript: rescript-react, bucklescript-tea

\subsection{Web-фреймворки}

dream \TODO

\subsection{Библиотеки для тестирования}

\TODO

\subsection{Библиотеки сериализации и десериализации}

Во многих языках программирования (де)сериализация построена на том, что
информация о типе значения доступна во время выполнения программы и с помощью рефлексии можно
получить например информацию о, например названии поля объекта и его значение и записать их в
сериализуемый результат. В OCaml, напротив, во время исполнения отсутствует всякое представление
о типе, т.е. например значение <<0>> типа <<int>> будет иметь то же самое представление в памяти что и
значение <<None>> типа <<option>> или значение <<false>> типа <<bool>> \cite{rwo-runtime-memory}.

Поэтому (де)сериализация в OCaml построена на другом принципе~-- должна быть определена функция которая
принимает сериализуемый объект и возвращает сериализованное представление (в случае сериализации),
или принимает сериализованное представление и возвращает десериализованный объект (в случае десериализации).
Такие функции программист может определить вручную, но это утомительно и при каждом структуры типа данных,
требуется изменять код функций. Решить эту проблему помогают препроцессорные расширения для OCaml, позволяющие
на этапе компиляции из определения типа данных сгенерировать функции для десериализации и сериализации.
Основные такие библиотеки, предоставляющие сериализованное представление и препроцессорные расширения,
являются Yojson (для JSON \cite{rwo-json}) и Sexplib (для S-выражений \cite{rwo-sexp}).

Использование S-выражений является более традиционным в функциональных языках программирования и их
уже используют выбранные библиотеки Core и Bonsai, поэтому для консистентности принято решение
не использовать JSON в проекте.

\subsection{Среда разработки и система сборки}

ocaml-lsp-server

\section{CSS-фреймворк}

Для оформления пользовательского интерфейса требуется CSS-фреймворк, обеспечивающий
отзывчивую вёрстку (responsive layout)~-- колонки, контейнеры и т.д.;
готовые компоненты~-- кнопки, таблицы, формы и т.д. Самым известным таким фреймворком является Bootstrap,
однако выбор сделан в пользу Bulma, который отличается оригинальностью и простотой \cite{bulma-vs-bootstrap}.

\section{Выбор библиотек и подходов}

Окончательный перечень используемых средств разработки таков:

\begin{itemize}
      \item OCaml~-- язык программирования;
      \item Js\_of\_ocaml~-- компилятор OCaml в JavaScript;
      \item Lwt~-- библиотека для конкурентного программирования;
      \item Core~-- стандартная библиотека;
      \item Dream~-- web-фреймворк;
      \item Sexplib~-- библиотека для сериализации и десериализации S-выражений;
      \item Bulma~-- CSS-фреймворк;
      \item VS Code~-- среда разработки;
      \item OCaml Platform~-- плагин для VS Code для работы с OCaml.
\end{itemize}
