\chapter{Проектирование}

\TODO

Кроме создания физического движка, важно продемонстрировать его работу. 
Желательно в интерактивном режиме. В XXI веке сложилась ситуация, что веб-приложения
обладают преимуществом, таким что для того чтобы им воспользовался пользователь, достаточно 
браузера и просто перейти по ссылке. 
Поэтому, решено сделать Single Page Application [\TODO], 
которое в интерактивном режиме отображает состояние модели, меняющееся с течением времени
в главе \TODO объяснено \TODO. 
 
Так как создаваемый движок будет способен работать в реальном времени, существует возможность
создать многопользовательский режим, для которого 



\section{Язык программирования}

\TODO

\section{Обзор экосистемы языка OCaml}

\TODO

\subsection{Компиляторы OCaml в JavaScript}

js\_of\_ocaml, rescript, ocamljs, (wasm \TODO) \TODO

\subsection{Стандартные библиотеки}

\TODO

ocaml: stdlib, base, core, containers, batteries, 

rescipt: belt, tablecloth

\subsection{ML-синтаксис и Reason-синтаксис}

\TODO

\subsection{Библиотеки конкурентного программирования}

lwt, async, eio \TODO

При построении программ, взаимодействующих с внешним миром,
часто приходиться ожидать: 
ожидать ответа от сервера, 
ожидать нажатия на кнопку, 

\subsection{Библиотеки для построения пользовательского интерфейса}

\TODO

js\_of\_ocaml: jsoo-react, ocaml-vdom, virtual\_dom, incr\_dom, bonsai

rescript: rescript-react, bucklescript-tea

\subsection{Библиотеки для RPC}

\TODO

\subsection{Web-фреймворки}

dream  \TODO

\subsection{Библиотеки для тестирования}

\TODO

\subsection{Библиотеки сериализации и десериализации}

\TODO

\textbf{Yojson}.

\textbf{Sexplib}.

\section{Прочие средства разработки}



\section{Выбор библиотек и подходов}

\TODO
