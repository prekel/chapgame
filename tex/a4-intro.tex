\Anonchapter{Введение}

Современные физические движки позволяют решать широкий круг задач,
от применения в компьютерных играх, до научного моделирования реальных физических процессов.
Однако, для симуляции простых ситуаций, например столкновение монет (или бильярдных шаров),
можно использовать не общее решение, пытающееся смоделировать как можно больше законов физики,
а разработать элегантное, формализованное и точное более частное решение.
В таком случае, следуя априорному подходу обнаружения столкновений,
можно свести такую частную задачу к решению алгебраических уравнений высокой степени,
для чего придётся применить численные методы.

\textbf{Целью выпускной квалификационной работы} является \Target.

Для достижения поставленной цели, были решены следующие задачи:

\Tasks

\clearpage
